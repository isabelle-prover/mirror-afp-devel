\documentclass[11pt,a4paper]{article}
\usepackage[T1]{fontenc}
\usepackage{isabelle,isabellesym}

% further packages required for unusual symbols (see also
% isabellesym.sty), use only when needed


\usepackage{amssymb,amsmath}
  %for \<leadsto>, \<box>, \<diamond>, \<sqsupset>, \<mho>, \<Join>,
  %\<lhd>, \<lesssim>, \<greatersim>, \<lessapprox>, \<greaterapprox>,
  %\<triangleq>, \<yen>, \<lozenge>

%\usepackage{eurosym}
  %for \<euro>

%\usepackage[only,bigsqcap]{stmaryrd}
%for \<Sqinter>

%\usepackage{eufrak}
  %for \<AA> ... \<ZZ>, \<aa> ... \<zz> (also included in amssymb)

%\usepackage{textcomp}
  %for \<onequarter>, \<onehalf>, \<threequarters>, \<degree>, \<cent>,
  %\<currency>

% this should be the last package used
\usepackage{pdfsetup}

% urls in roman style, theory text in math-similar italics
\urlstyle{rm}
\isabellestyle{it}


% for uniform font size
%\renewcommand{\isastyle}{\isastyleminor}


\begin{document}

\title{No-free-lunch theorem for machine learning}
\author{Michikazu Hirata}
\maketitle
\begin{abstract}
  This entry is a formalization of the no-free-lunch theorem
  for machine learning following Section~5.1 of the book
  \textit{Understanding Machine Learning: From Theory to Algorithms}~\cite{shalev2014}
  by Shai Shalev-Shwartz and Shai Ben-David.
  The theorem states that for binary classification prediction tasks,
  there is no universal learner, meaning that
  for every learning algorithms, there exists a distribution
  on which it fails.
\end{abstract}

\tableofcontents

% sane default for proof documents
\parindent 0pt\parskip 0.5ex

% generated text of all theories
\input{session}

% optional bibliography
\bibliographystyle{abbrv}
\bibliography{root}

\end{document}

%%% Local Variables:
%%% mode: latex
%%% TeX-master: t
%%% End:
