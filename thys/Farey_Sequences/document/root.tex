\documentclass[11pt,a4paper]{article}
\usepackage[T1]{fontenc}
\usepackage{isabelle,isabellesym}
\usepackage{amsmath,amssymb}

% this should be the last package used
\usepackage{pdfsetup}

% urls in roman style, theory text in math-similar italics
\urlstyle{rm}
\isabellestyle{it}

%%Euro-style date: 20 September 1955
\def\today{\number\day\space\ifcase\month\or
January\or February\or March\or April\or May\or June\or
July\or August\or September\or October\or November\or December\fi
\space\number\year}

\begin{document}

\title{Farey Sequences and Ford Circles}
\author{Lawrence C. Paulson}
\maketitle

\begin{abstract}

The sequence $F_n$ of \emph{Farey fractions} of order~$n$ 
has the form 
$$\frac{0}{1}, \frac{1}{n}, \frac{1}{n-1}, \ldots, \frac{n-1}{n}, \frac{1}{1}$$
where the fractions appear in numerical order and have denominators at most $n$.
The transformation from $F_n$ to $F_{n+1}$ can be effected by combining adjacent elements of 
the sequence~$F_n$, using an operation called the \emph{mediant}.
Adjacent (reduced) fractions $(a/b) < (c/d)$ satisfy the \emph{unimodular}
relation $bc - ad = 1$ and their mediant is $\frac{a+c}{b+d}$.
A \emph{Ford circle} is specified by a rational number, and interesting consequences follow
in the case of Ford circles obtained from some Farey sequence~$F_n$.
The formalised material is drawn from Apostol's \emph{Modular Functions and Dirichlet Series in Number Theory}
\cite{apostol-modular-functions}.

\end{abstract}

\newpage

% include generated text of all theories
\input{session}

\paragraph*{Acknowledgements}
Manual Eberl set up the initial Farey development.

\bibliographystyle{abbrv}
\bibliography{root}

\end{document}
