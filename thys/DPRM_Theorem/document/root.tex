\documentclass[11pt,a4paper]{article}
\usepackage[T1]{fontenc}
\usepackage{isabelle,isabellesym}
\usepackage{authblk}

% further packages required for unusual symbols (see also
% isabellesym.sty), use only when needed

%\usepackage{amssymb}
  %for \<leadsto>, \<box>, \<diamond>, \<sqsupset>, \<mho>, \<Join>,
  %\<lhd>, \<lesssim>, \<greatersim>, \<lessapprox>, \<greaterapprox>,
  %\<triangleq>, \<yen>, \<lozenge>

%\usepackage{eurosym}
  %for \<euro>

%\usepackage[only,bigsqcap]{stmaryrd}
  %for \<Sqinter>

%\usepackage{eufrak}
  %for \<AA> ... \<ZZ>, \<aa> ... \<zz> (also included in amssymb)

%\usepackage{textcomp}
  %for \<onequarter>, \<onehalf>, \<threequarters>, \<degree>, \<cent>,
  %\<currency>

% this should be the last package used
\usepackage{pdfsetup}

% urls in roman style, theory text in math-similar italics
\urlstyle{rm}
\isabellestyle{it}

% for uniform font size
%\renewcommand{\isastyle}{\isastyleminor}


\begin{document}

\title{Diophantine Equations and the DPRM Theorem}

\author{Jonas Bayer\thanks{Equal contribution.} \hspace*{10ex} Marco David\textsuperscript{*}
         \hspace*{10ex} Benedikt Stock\textsuperscript{*} \\ Abhik Pal
         \hspace*{10ex} Yuri Matiyasevich\thanks{Contributed by supplying a detailed proof and an
        initial introduction to Isabelle.}
         \hspace*{10ex} Dierk Schleicher}

\maketitle

%\footnotetext[*]{Text}

\begin{abstract}
	We present a formalization of Matiyasevich's proof of the DPRM theorem, which states that every
  recursively enumerable set of natural numbers is Diophantine. This result from 1970 yields a
  negative solution to Hilbert's 10th problem over the integers. To represent recursively
  enumerable sets in equations, we implement and arithmetize register machines. We formalize a
  general theory of Diophantine sets and relations to reason about them abstractly.
  Using several number-theoretic lemmas, we prove that exponentiation has a Diophantine
  representation.
\end{abstract}

\tableofcontents

\newpage

% sane default for proof documents
\parindent 0pt\parskip 0.5ex

% Some preliminary text
\paragraph{Overview}
A previous short paper~\cite{dprm_isabelle} gives an overview
of the formalization. In particular, the challenges of implementing the notion of 
diophantine predicates is discussed and a formal definition of register machines is described.
Another meta-publication~\cite{cicm} recounts our learning experience throughout this project.

The present formalisation is based on Yuri Matiyasevich's monograph~\cite{h10lecturenotes} which
contains a full proof of the DPRM theorem. This result or parts of its proof have also been
formalized in other interactive theorem provers, notably in Coq~\cite{dprm_coq},
Lean~\cite{dprm_lean} and Mizar~\cite{dprm_mizar1,dprm_mizar2}.

\paragraph{Acknowledgements}
We want to thank everyone who participated in the formalization
during the early stages of this project:
Deepak Aryal, Bogdan Ciurezu, Yiping Deng, Prabhat Devkota,
Simon Dubischar, Malte Haßler, Yufei Liu and Maria Antonia Oprea. Moreover, we would like to
express our sincere gratitude to the entire welcoming and supportive Isabelle
community. In particular, we are indebted to Christoph Benzmüller for his
expertise, his advice and for connecting us with relevant experts in the field.
Among those, we specially want to thank Mathias Fleury for all his help with Isabelle.
Finally, we would like to thank the DFG for supporting our attendance at several events and conferences, allowing us to present the project to a broad audience.

\newpage

% generated text of all theories
\input{session}

% optional bibliography
\bibliographystyle{abbrv}
\bibliography{root}

\end{document}

%%% Local Variables:
%%% mode: latex
%%% TeX-master: t
%%% End:
