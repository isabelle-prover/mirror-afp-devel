\documentclass[11pt,a4paper]{article}
\usepackage[T1]{fontenc}
\usepackage{isabelle,isabellesym}

% further packages required for unusual symbols (see also
% isabellesym.sty), use only when needed

\usepackage{amssymb}

%\usepackage[only,bigsqcap,bigparallel,fatsemi,interleave,sslash]{stmaryrd}
  %for \<Sqinter>, \<Parallel>, \<Zsemi>, \<Parallel>, \<sslash>

% this should be the last package used
\usepackage{pdfsetup}

% urls in roman style, theory text in math-similar italics
\urlstyle{rm}
\isabellestyle{literal}

\begin{document}

\title{Greibach Normal Form}
\author{Alexander Haberl and Tobias Nipkow and Akihisa Yamada}
\maketitle

\begin{abstract}
  This theory formalizes Hopcroft and Ullman's algorithm
  \cite{HopcroftU79} to transform a set of productions into Greibach
  Normal Form (GNF) \cite{Greibach}. We concentrate on the essential
  property of the GNF: every production starts with a terminal; the
  tail of a rhs may contain further terminals.  The complexity of the
  algorithm can be exponential.
\end{abstract}

% sane default for proof documents
\parindent 0pt\parskip 0.5ex

% generated text of all theories
\input{session}

\bibliographystyle{abbrv}
\bibliography{root}

\end{document}

%%% Local Variables:
%%% mode: latex
%%% TeX-master: t
%%% End:
