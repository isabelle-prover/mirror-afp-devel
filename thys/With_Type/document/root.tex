\documentclass[11pt,a4paper]{article}
\usepackage[utf8]{inputenc}
\usepackage[T1]{fontenc}
\usepackage{geometry}
\usepackage{isabelle,isabellesym}
\usepackage{amsmath, amssymb}
\usepackage{mathpartir}
\usepackage{newunicodechar}
\usepackage{pdfsetup}

% urls in roman style, theory text in math-similar italics
\urlstyle{rm}
\isabellestyle{it}

\newunicodechar{∀}{\ensuremath\forall}
\newunicodechar{∃}{\ensuremath\exists}

% \DeclareUnicodeCharacter{2200}{\ensuremath\forall}
% \DeclareUnicodeCharacter{2203}{\ensuremath\exists}


\begin{document}

\title{With-Type -- Poor man's dependent types\thanks{Supported by the ERC consolidator grant CerQuS (819317), the PRG team grant “Secure Quantum Technology” (PRG946) from the Estonian Research Council, and the Estonian Cluster of Excellence ``Foundations of the Universe'' (TK202).}}
\author{Dominique Unruh\\
 \footnotesize RWTH Aachen, University of Tartu}
\maketitle

\begin{abstract}
  The type system of Isabelle/HOL does not support dependent types or arbitrary quantification over types.
  We introduce a system to mimic dependent types and existential quantification over types \emph{in limited circumstances} at the top level of theorems.
\end{abstract}

\tableofcontents

% sane default for proof documents
\parindent 0pt\parskip 0.5ex

\section{Introduction}

The type system of Isabelle/HOL is relatively limited when it comes to the treatment of types (at least when compared with systems such as Coq or Lean).
There is no support for arbitrary quantification over types, nor can types depend on other values.
\emph{Universal} quantification over types is implicitly possible at the top level of a theorem because type variables are treated as universally quantified.%
\footnote{For example, a theorem such as \texttt{(1::?'a) + 1 = 2} can be interpreted as \texttt{∀a.~(1::a) + 1 = 2}.}
In a very limited way, we can also mimic existential quantification on the top level:
Instead of saying, e.g., \texttt{∃a.~card (UNIV ::~a set) = 3} (``there exists a type with three elements''),
we can define a type with the desired property (\texttt{typedef witness = "{1,2,3::int}"}) and prove \texttt{card (UNIV ::~witness set) = 3}. This achieves the same thing but it suffers from several drawbacks:
\begin{itemize}
\item We can only use this encoding at the top level of theorems. E.g., we cannot represent the claim \texttt{P (∃a.~card (UNIV ::~a set) = 3)} where \texttt{P} is an arbitrary predicate.
\item It only works when we can explicitly construct the type that is claimed to exist (because we need to describe it in the \texttt{typedef} command).
\item The witness we give cannot depend on variables local to the current theorem or proof because the \texttt{typedef} command can only be given on the top level of a theory, and can only depend on constants.
  E.g., it would not be possible to express something like:
  \begin{equation}\label{eq:ex.any.size}
    \texttt{∀n::nat. (n >= 1 ---> (∃a.~card (UNIV ::~a set) = n))}.
  \end{equation}
\end{itemize}
In this work, we resolve the third limitation.
Concretely, we will be able to define a set (not a type!) \texttt{witness n} that depends on a natural number \texttt{n},\footnote{In this example, \texttt{witness n} simply has to be an arbitrary \texttt{n}-element set, e.g., \texttt{witness n = \{..<n\}}.}
and write:
\[
\texttt{n >= 1 ---> let 'a::type=witness n in (card (UNIV ::~'a set) = n)} 
\]
This statement is read as:
\begin{quote}
  If \texttt{n >= 1}, and \texttt{'a} is defined to be the type described by the set \texttt{witness n}
  (imagine a \emph{local} \texttt{typedef 'a = "witness n"}), then \texttt{card (UNIV ::~a set) = n} holds.
\end{quote}
This is nothing else than~\eqref{eq:ex.any.size} with an explicitly specified witness.

We call the Isabelle constant implementing this construct \texttt{with\_type}, because \texttt{let 'a::type=witness in P} can be read as ``with type \texttt{'a} defined by \texttt{witness}, \texttt{P} holds''.

Since in \texttt{let 'a::type=spec in \dots}, the \texttt{spec} can depend on local variables,
we essentially have encoded a limited version of dependent types.
Limited because our encoding is not meaningful except at the top level of a theorem (``\texttt{premises ==> let 'a::type = \dots}'' is ok, ``\texttt{P (let 'a::type = \dots)}'' for arbitrary \texttt{P} is not).

To be able to actually use this encoding in proofs, we implement three reasoning rules for introduction, elimination, and modus ponens.
These are \emph{roughly} the following:
\begin{center}
  $\inferrule*[right=Intro]{
    \texttt{P}\ \textit{(given typedef)}
  }{
    \texttt{let 'a::type = w in P}
  }$

  \medskip
  
  $\inferrule*[right=Elim]{
    \texttt{let 'a::type = w in P}
    \\
    \texttt{'a}\textit{ does not occur in }\texttt{P}
  }{
    \mathtt{P}
  }$

  \medskip
  
  $\inferrule*[right=ModusPonens]{
    \texttt{let 'a::type = w in P}
    \\
    \texttt{P ==> Q}\ \textit{(given typedef)}
  }{
    \texttt{let 'a::type = w in Q}
  }$
\end{center}
Here ``\textit{(given typedef)}'' means that the respective premise can be shown in a context where
the local equivalent of a \texttt{typedef 'a = "w"} was declared.
(In particular, there are morphisms \texttt{rep}, \texttt{abs} between \texttt{'a} and the set \texttt{w}.)

The elimination rule uses the \texttt{Types\_To\_Sets} extension \cite{types-to-sets} to get rid of the ``unused'' \texttt{let 'a::type}.

\bigskip

The \texttt{with\_type} mechanism is not limited to types of type class \texttt{type}
(the Isabelle/HOL type class containing all types).
We can also write, e.g., \texttt{let 'a::ab\_group\_add = set with ops in P} which would say that \texttt{'a} is an abelian additive group (type class \texttt{ab\_group\_add}) defined via \texttt{typedef 'a = "set"} with group operations \texttt{ops} (which specifies the addition operation, the neutral element, etc.).


% generated text of all theories
\input{session}

% optional bibliography
\bibliographystyle{abbrv}
\bibliography{root}

\end{document}

%%% Local Variables:
%%% mode: latex
%%% TeX-master: t
%%% End:


