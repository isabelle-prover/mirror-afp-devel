\documentclass[11pt,a4paper]{article}
\usepackage[T1]{fontenc}
\usepackage{isabelle,isabellesym}
\usepackage{amsfonts,amsmath, amssymb,amsthm}
\usepackage[top=1.4in, bottom=1.4in, left=1.4in, right=1.4in]{geometry}

\usepackage[only,bigsqcap]{stmaryrd}

\usepackage{marginnote}
\usepackage{algorithm}
\usepackage[noend]{algpseudocode}
\renewcommand{\algorithmicrequire}{\textbf{Input:}}
\renewcommand{\algorithmicensure}{\textbf{Output:}}
\let\oldsection\section
\renewcommand\section{\clearpage\oldsection}

\newcommand{\getsr}{\xleftarrow{\$}}

\usepackage{mathtools}
\DeclarePairedDelimiter{\ceil}{\lceil}{\rceil}
\DeclareMathOperator{\prob}{\mathcal P}
\DeclareMathOperator{\expect}{\mathbb E}
\DeclareMathOperator{\Ber}{\mathrm{Ber}}
\DeclareMathOperator{\indicator}{\mathrm{I}}

\newcommand{\eps}{\varepsilon}
\renewcommand{\phi}{\varphi}
\newtheorem{theorem}{Theorem}
\theoremstyle{definition}
\newtheorem{lemma}{Lemma}

% this should be the last package used
\usepackage{pdfsetup}

% urls in roman style, theory text in math-similar italics
\urlstyle{rm}
\isabellestyle{it}

\begin{document}

\title{Verification of the CVM algorithm with a New Recursive Analysis Technique}
\author{Emin Karayel, Derek Khu, Kuldeep S. Meel, Yong Kiam Tan,\\and Seng Joe Watt}
\maketitle

\begin{abstract}
In 2022, Chakraborty et al.~\cite{chakraborty2022} published a streaming
algorithm (henceforth, the CVM algorithm) for the distinct
elements problem, that deviated considerably from the state-of-the art, due to its simplicity
and avoidance of standard derandomization techniques, while still maintaining a close to optimal
logarithmic space complexity.

In this entry, we verify the CVM algorithm's correctness using a new technique which simplifies
the analysis considerably compared to the orignal proof by Chakraborty et al. The main idea is
based on a probabilistic invariant that allows us to derive concentration bounds using the
Cram\'{e}r--Chernoff method.

This new technique opens up the possible algorithm design space, and we introduce a new variant of
the CVM algorithm, that is total, and also has an additional property in addition to concentration:
unbiasedness. This means the expected result of the algorithm is exactly equal to the desired
result. The latter is also a new property, that neither the original CVM algorithm nor classic
algorithms for the distinct elements problem possess.
\end{abstract}

\tableofcontents

\parindent 0pt\parskip 0.5ex

\input{session}

\bibliographystyle{abbrv}
\bibliography{root}

\appendix
\section{Informal Proof\label{apx:informal_proof}}
This section includes an informal version of the proof for the tail bounds and unbiasedness
of the abstract algorithm (Algorithm~\ref{alg:cvm_abs}) for interested readers.

This means we assume the $\mathrm{subsample}(\chi)$ operation fulfills
Eq.~\ref{eq:subsample_condition} and always returns a subset of $\chi$.

\paragraph{Notation:} For a finite set $S$, the probability space of uniformly sampling from the set
is denoted by $U(S)$, i.e., for each $s \in S$ we have $\prob_{U(S)}({s}) = |S|^{-1}$.
We write $\mathrm{Ber}(p)$ for the Bernoulli probability space, over the set $\{0,1\}$, i.e.,
$P_{\mathrm{Ber}(p)}(\{1\}) = p$. $I(P)$ is the indicator function for a predicate $P$, i.e.,
$I(\mathrm{true}) = 1$ and $I(\mathrm{false}) = 0$.

Like in the formalization, we will denote the first five lines of the loop ($3$--$7$) as step 1,
the last four lines ($8$--$10$) as step 2. For the distribution of the state of the algorithm after
processing $i$ elements of the sequence, we will write $\Omega_i$.
The elements of the probability spaces are pairs composed of a set and the number of subsampling
steps, representing $\chi$ and $p$ respectively.

For example: $\Omega_0 = U(\{(\emptyset, 1)\})$ is the initial state,
$\Omega_1 = U(\{(\{a_1\}, 1)\})$, etc., and $\Omega_l$ denotes the final state.
We introduce $\chi$ and $p$ as random variables defined over such probability spaces $\Omega$,
in particular, $\chi$ (resp. $p$) is the projection to the first (resp. second) component.

The state of the algorithm after processing only step 1 of the $i$-th loop iteration is denoted by
$\Omega'_i$. So the sequence of states is represented by the distributions
$\Omega_0, \Omega'_1, \Omega_1, \ldots, \Omega'_l, \Omega_l$.

\subsection{Loop Invariant}
After these preliminaries, we can go to the main proof, whose core is a probabilistic loop
invariant for Algorithm~\ref{alg:cvm_abs} that can be verified inductively.

\begin{lemma}
\label{le:prob_invariant}
Let $\phi : (0,1] \times \{0,1\} \rightarrow \mathbb R_{\geq 0}$ be a function, fulfilling the following
  conditions:
\begin{enumerate}
\item \label{cond:phi_1} $(1-\alpha) \phi(x,0) + \alpha \phi(x,1) \leq \phi(x/\alpha,1)$ for all
  $0 < \alpha < 1$, $0 < x \leq 1$, and
\item \label{cond:phi_2} $\phi(x,0) \leq \phi(y,0)$ for all $0 < x < y \leq 1$.
\end{enumerate}
Then for all
$k \in \{0,\ldots,l\}$, $S \subseteq \{a_1,..,a_k\}$, $\Omega \in \{\Omega_k,\Omega_k'\}$:
\[
  \expect_{\Omega}\left[ \prod_{s \in S} \phi(p, I(s \in \chi)) \right] \leq \phi(1,1)^{|S|}
\]
\end{lemma}
\begin{proof}
We show the result using induction over $k$. Note that we show the statement for arbitrary $S$,
i.e., the induction statements are:
\begin{eqnarray*}
P(k) & :\leftrightarrow & \left(\forall S \subseteq \{a_1,..,a_k\}. \;
  \expect_{\Omega_k}\left[ \prod_{s \in S} \phi(p, I(s \in \chi)) \right]
    \leq \phi(1,1)^{|S|} \right) \\
Q(k) & :\leftrightarrow & \left(\forall S \subseteq \{a_1,..,a_k\}. \;
  \expect_{\Omega'_k}\left[ \prod_{s \in S} \phi(p, I(s \in \chi)) \right]
    \leq \phi(1,1)^{|S|} \right)
\end{eqnarray*}
and we will show $P(0),Q(1),P(1),Q(2),P(2),\ldots,Q(l),P(l)$ successively.
\paragraph{Induction start $P(0)$:} \phantom{.}\\
We have $S \subseteq \emptyset$, and hence
\[
\expect_{\Omega_0}\left[ \prod_{s \in S} \phi(p, I(s \in \chi)) \right] =
\expect_{\Omega_0}\left[1\right] = 1 \leq \phi(1,1)^0 \textrm{.}
\]
\paragraph{Induction step $P(k) \rightarrow Q(k+1)$:} \phantom{.}\\
Let $S \subseteq \{ a_1, \ldots, a_{k+1} \}$ and define $S' := S - \{ a_{k+1} \}$.
Note that $\Omega'_{k+1}$ can be constructed from $\Omega_k$ as a compound distribution, where
$a_{k+1}$ is included in the buffer, with the probability $p$, which is itself a random variable
over the space $\Omega_k$.

In particular, for example:
\[
\prob_{\Omega'_{k+1}}( P(\chi, p) ) = \int_{\Omega_k} \int_{\textrm{Ber}(p(\omega))}
P(\chi(\omega)-\{a_{k+1}\}\cup f(\tau), p(\omega)) \, d \tau \, d \omega
\]
for all predicates $P$ where we define $f(1) = \{a_{k+1}\}$ and $f(0) = \emptyset$.

We distinguish the two cases $a_{k+1} \in S$ and $a_{k+1} \notin S$. If $a_{k+1} \in S$:
\begin{eqnarray*}
  & & \textstyle \expect_{\Omega'_{k+1}}\left[\prod_{s \in S} \phi(p, I(s \in \chi)) \right] \\
 & = & \textstyle \int_{\Omega_k} \left(\prod_{s \in S'} \phi(p, I(s \in \chi))\right)
  \int_{\mathrm{Ber}(p(\omega))} \phi(p, \tau) \, d \tau \, d \omega \\
  & = & \textstyle \int_{\Omega_k} \left(\prod_{s \in S'} \phi(p, I(s \in \chi))\right) ((1-p) \phi(p,0) +
  p \phi(p,1)) \, d \omega\\
  & \underset{\textrm{Cond~\ref{cond:phi_1}}}{\leq} & \textstyle \int_{\Omega_k} \left( \prod_{s \in S'}
  \phi(p, I(s \in \chi)) \right) \phi(1,1) \, d \omega \\
  & \underset{\textrm{IH}}{\leq} & \textstyle \phi(1,1)^{|S'|} \phi(1,1) = \phi(1,1)^{|S|}
\end{eqnarray*}

If $a_{k+1} \notin S$ then $S' = S$ and:
\[
  \textstyle \expect_{\Omega'_{k+1}}\left[\prod_{s \in S} \phi(p, I(s \in \chi)) \right] =
  \int_{\Omega_k} \prod_{s \in S} \phi(p, I(s \in \chi)) \, d \omega \leq_{\textrm{IH}} \phi(1,1)^{|S'|} =
  \phi(1,1)^{|S|}
\]
\paragraph{Induction step $Q(k+1) \rightarrow P(k+1)$:} \phantom{.}\\
Let $S \subseteq \{ a_1, \ldots, a_{k+1} \}$.

Let us again note that $\Omega_{k+1}$ is a compound distribution over $\Omega'_{k+1}$. In general, for
all predicates $P$:
\begin{align*}
  & \prob_{\Omega_{k+1}}( P(\chi, p) ) = \\
  & \int_{\Omega'_{k+1}} I( |\chi(\omega)|<n ) P(\chi(\omega),p(\omega)) +
  I( |\chi(\omega)|=n ) \int_{\mathrm{subsample}(\chi(\omega))}
  P(\tau, f p(\omega)) \, d \tau \, d \omega \textrm{.}
\end{align*}

With this we can can now verify the induction step:
\begin{eqnarray*}
&  & \textstyle\expect_{\Omega_{k+1}}\left[\prod_{s \in S} \phi(p, I(s \in \chi))\right]  \\
& = & \textstyle \int_{\Omega'_{k+1}} I(|\chi|<n)\prod_{s\in S} \phi(p, I(s \in \chi)) \, d \omega\\
& + & \textstyle \int_{\Omega'_{k+1}} I(|\chi|=n)
  \prod_{s \in S \setminus \chi(\omega)} \phi(pf,0)
    \int_{\mathrm{subsample}(\chi)} \prod_{s \in S \cap \chi} \phi(pf,I(s \in \tau )) d \tau \, d \omega \\
& \leq \marginnote{\footnotesize Eq.~\ref{eq:subsample_condition}} & \textstyle \int_{\Omega'_{k+1}}
  I(|\chi| < n)\prod_{s \in S} \phi(p, I(s \in \chi)) \, d \omega \\
& + & \textstyle \int_{\Omega'_{k+1}} I(|\chi| = n) \prod_{s \in S \setminus \chi(\omega)}
  \phi(pf,0) \prod_{s \in S \cap \chi} \int_{\mathrm{Ber(f)}} \phi(pf,\tau) d \tau \, d \omega  \\
& \leq \marginnote{\footnotesize Cond~\ref{cond:phi_2}}  &
  \textstyle \int_{\Omega'_{k+1}} I(|\chi| < n)\prod_{s \in S} \phi(p, I(s \in \chi)) \, d \omega \\
& + & \textstyle \int_{\Omega'_{k+1}} I(|\chi| = n) \prod_{s \in S \setminus \chi(\omega)}
  \phi(p,0) \prod_{s \in S \cap \chi} ((1-f)\phi(pf,0)+f\phi(pf,1)) \, d \omega  \\
& \leq \marginnote{\footnotesize Cond~\ref{cond:phi_1}} & \textstyle \int_{\Omega'_{k+1}} I(|\chi| < n)
  \prod_{s \in S} \phi(p, I(s \in \chi)) \, d \omega \\
& + & \textstyle \int_{\Omega'_{k+1}} I(|\chi| = n) \prod_{s \in S \setminus \chi(\omega)}
  \phi(p,0) \prod_{s \in S \cap \chi} \phi(p,1) \, d \omega \\
& = & \textstyle \int_{\Omega'_{k+1}} I(|\chi|<n)\prod_{s \in S} \phi(p,I(s\in\chi)) \, d\omega \\
& + & \textstyle \int_{\Omega'_{k+1}} I(|\chi|=n)\prod_{s \in S} \phi(p,I(s\in\chi)) \, d\omega \\
& = & \textstyle \expect_{\Omega'_{k+1}}\left[\prod_{s \in S} \phi(p, I(s \in \chi)) \right]
  \leq  \marginnote{\footnotesize IH} \phi(1,1)^{|S|}
\end{eqnarray*}
\end{proof}

A corollary and more practical version of the previous lemma is:
\begin{lemma}\label{le:prob_invariant_simple}
Let $q \leq 1$ and $h : \mathbb [0,q^{-1}] \rightarrow \mathbb R_{\geq 0}$ be concave then for all
$k \in \{0,\ldots,l\}$, $S \subseteq \{a_1,..,a_k\}$, $\Omega \in \{\Omega_k,\Omega_k'\}$:
\[
  \expect_{\Omega}\left[ \prod_{s \in S} I(p>q) h(p^{-1} I(s \in \chi)) \right] \leq h(1)^{|S|}
\]
\end{lemma}
\begin{proof}
Follows from Lemma~\ref{le:prob_invariant} for $\phi(p,\tau) := I(p > q) h(\tau p^{-1})$. We just
need to check Conditions 1 and 2. Indeed,
\begin{eqnarray*}
  (1-\alpha) \phi(x,0) + \alpha \phi(x,1) & = & (1-\alpha) I(x>q) h(0) + \alpha I(x>q) h(x^{-1}) \\
   & \leq & I(x > q) h(\alpha x^{-1}) \leq I(x > q \alpha) h(\alpha x^{-1}) = \phi(x/\alpha,1)
\end{eqnarray*}
for $0 < \alpha < 1$ and $0< x \leq 1$, where we used that $x > q$ implies $x > q \alpha$; and
\[
  \phi(x,0) = I(x>q) h(0) \leq I(y > q) h(0) = \phi(y,0)
\]
for $0 < x < y \leq 1$, where we used that $x > q$ implies $y > q$.
\end{proof}

It should be noted that this is a probabilistic recurrence relation, but the main innovation is that
we establish a relation, with respect to general classes of functions of the state variables.

\subsection{Concentration}
Let us now see how we can obtain concentration bounds using Lemma~\ref{le:prob_invariant_simple}, i.e., that
the result of the algorithm is concentrated around the cardinality of $A = \{ a_1, \ldots, a_l \}$.
This will be done using the Cram\'{e}r--Chernoff method for the probability that the estimate is
above $(1+\eps)|A|$ (resp.~below $(1-\eps)|A|$) assuming $p$ is not too small and a tail estimate
for $p$ being too small.

It should be noted that concentration is trivial, if $|A| < n$, i.e., if we never need to do
sub-sampling, so we assume $|A| \geq n$.

Define $q := n/(4|A|)$ and notice that $q \leq \frac{1}{4}$.

Let us start with the upper tail bound:
\begin{lemma}
\label{le:upper_tail}
For any $\Omega \in \{\Omega_0,\ldots,\Omega_l\} \cup \{\Omega'_1,\ldots,\Omega'_l\}$ and
$0 < \eps \leq 1$:
\[
L := \prob_{\Omega} \left( p^{-1} |\chi| \geq (1+\eps) |A| \wedge p \geq q\right) \leq
  \exp\left(-\frac{n}{12} \eps^2\right)
\]
\end{lemma}
\begin{proof}
By assumption there exists a $k$ such that $\Omega \in \{\Omega_k, \Omega'_k\}$.
Let $A' = A \cap \{a_1,\ldots,a_k\}$.
Moreover, we define:
\begin{align*}
  t & := q \ln( 1 + \eps) \\
  h(x) & := 1+qx(e^{t/q}-1)
\end{align*}

To get a tail estimate, we use the Cram\'{e}r--Chernoff method:
{\allowdisplaybreaks
\begin{eqnarray*}
  L & \underset{t > 0}{\leq} & \prob_{\Omega} \left( \exp(t p^{-1} |\chi|) \geq
    \exp(t (1+\eps) |A|) \wedge p \geq q \right) \\
  & \leq &
    \prob_{\Omega} \left( I(p \geq q) \exp(t p^{-1} |\chi|) \geq \exp(t (1+\eps) |A|) \right) \\
  & \underset{\textrm{Markov}}{\leq} &
    \exp( -t (1+\eps) |A| ) \expect_\Omega \left[ I(p \geq q) \exp ( t p^{-1} |\chi| ) \right] \\
 & \leq & \exp( -t (1+\eps) |A| )
  \expect_\Omega \left[ \prod_{s \in A'} I(p \geq q) \exp ( t p^{-1} I(s \in \chi) ) \right] \\
 & \leq & \exp( -t (1+\eps) |A| )
  \expect_\Omega \left[ \prod_{s \in A'} I(p \geq q) h( p^{-1} I(s \in \chi) ) \right] \\
 & \underset{\textrm{Le.~\ref{le:prob_invariant_simple}}}{\leq} & \exp( -t (1+\eps) |A| ) h(1)^{|A'|} \\
 & \underset{h(1)\geq 1}{\leq} & \left(\exp( \ln(h(1)) -t (1+\eps))\right) ^ {|A|}
\end{eqnarray*}}
So we just need to show that (using $|A|=\frac{n}{4q}$):
\[
  \ln(h(1))-t (1+\eps) \leq \frac{-q \eps^2}{3}
\]
The latter can be established by analyzing the function
\[
  f(\eps) := -\ln(1+q \eps) + q \ln(1+\eps) (1+\eps) - \frac{q \eps^2}{3} =
    -\ln (h(1)) +t (1+\eps) - \frac{q\eps^2}{3} \textrm{.}
\]
For which it is easy to check $f(0) = 0$ and the derivative with respect to $\eps$ is non-negative
in the range $0 \leq q \leq 1/4$ and $0 < \eps \leq 1$, i.e., $f(\eps) \geq 0$.
\end{proof}

Using the previous result we can also estimate bounds for $p$ becoming too small:
\begin{lemma}\label{le:low_p}
\[
\prob_{\Omega_l}(p < q) \leq l \exp\left( - \frac{n}{12}\right)
\]
\end{lemma}
\begin{proof}
We will use a similar strategy as in the $\mathrm{Bad}_2$ bound from the original CVM paper~\cite{chakraborty2022}.
Let $j$ be maximal, s.t., $q \leq f^j$. Hence $f^{j+1} < q$ and:
\begin{equation}
\label{eq:f_j}
  f^j \leq 2f f^j < 2q = \frac{n}{2|A|} \textrm{.}
\end{equation}
First, we bound the probability of jumping from $p=f^j$ to $p=f^{j+1}$ at a specific point in the
algorithm, e.g., while processing $k$ stream elements. It can only happen if $|\chi|=n$, $p=f^j$
in $\Omega'_k$. Then
\begin{eqnarray*}
  \prob_{\Omega'_k} ( |\chi| \geq n \wedge p=f^j) & \leq &
    \prob( p^{-1} |\chi| \geq f^{-j} n \wedge p \geq q) \\
    & \underset{\textrm{Eq.~\ref{eq:f_j}}}{\leq} & \prob( p^{-1} |\chi| \geq 2|A| \wedge p \geq q)\\
    & \underset{\textrm{Le.~\ref{le:upper_tail}}}{\leq} & \exp(- n/12)
\end{eqnarray*}
The probability that this happens ever in the entire process is then at most $l$ times the above
which proves the lemma.
\end{proof}

\begin{lemma}\label{le:lower_tail}
Let $0 < \eps < 1$ then:
\[
  L := \prob_{\Omega_l} ( p^{-1} |\chi| \leq (1-\eps)|A| \wedge p \geq q) \leq
    \exp\left(-\frac{n}{8} \eps^2\right)
\]
\end{lemma}
\begin{proof}
Let us define
\begin{align*}
  t & := q \ln(1-\eps) < 0 \\
  h(x) & := 1+q x (e^{t/q}-1)
\end{align*}

Note that $h(x) \geq 0$ for $0 \leq x \leq q^{-1}$ (can be checked by verifying it is true for
$h(0)$ and $h(q^{-1})$ and the fact that the function is affine.)

With these definitions we again follow the Cram\'{e}r--Chernoff method:
{\allowdisplaybreaks
\begin{eqnarray*}
  L & \underset{t<0}{=} & \prob_{\Omega_l} \left( \exp(t p^{-1} |\chi|) \geq
    \exp(t (1-\eps)|A|) \wedge p \geq q \right) \\
    & \leq & \prob_{\Omega_l} \left( I(p \geq q) \exp(t p^{-1} |\chi|) \geq
    \exp(t (1-\eps)|A|) \wedge p > q \right) \\
    & \underset{\textrm{Markov}}{\leq} & \exp( -t (1-\eps) |A| )
    \expect_\Omega \left[ I(p \geq q) \exp ( t p^{-1} |\chi| ) \right] \\
    & = & \exp( -t (1-\eps) |A| ) \expect_\Omega \left[ \prod_{s \in A} I(p \geq q)
      \exp ( t p^{-1} I(s \in \chi) ) \right] \\
    & \leq & \exp( -t (1-\eps) |A| ) \expect_\Omega
      \left[ \prod_{s \in A} I(p \geq q) h( p^{-1} I(s \in \chi) ) \right] \\
    & \underset{\textrm{Le.~\ref{le:prob_invariant_simple}}}{\leq} & \exp( -t (1-\eps) |A| ) (h(1))^{|A|} \\
    & = & \exp( \ln(h(1)) -t (1-\eps) )^{|A|}
\end{eqnarray*}
}
Substituting $t$ and $h$ and using $|A| = \frac{n}{4q}$, we can see that the lemma is true if
\[
  f(\eps) := q \ln(1-\eps) (1-\eps) -\ln(1 - q \eps) - \frac{q}{2}\eps^2 =
  t (1-\eps) - \ln(h(1)) - \frac{q}{2} \eps^2
\]
is non-negative for $0 \leq q \leq \frac{1}{4}$ and $0 < \eps < 1$.
This can be verified by checking that $f(0) = 0$ and that the derivative with respect to $\eps$ is
non-negative.
\end{proof}
We can now establish the concentration result:
\begin{theorem}\label{th:concentration}
Let $0 < \eps < 1$ and $0 < \delta < 1$ and
$n \geq \frac{12}{\eps^2} \ln\left(\frac{3l}{\delta}\right)$ then:
\[
  L = \prob_{\Omega_l} \left( | p^{-1} |\chi| - |A| | \geq \eps |A| \right) \leq \delta
\]
\end{theorem}
\begin{proof}
Note that the theorem is trivial if $|A| < n$. If not:
\begin{eqnarray*}
  L & \leq & \prob_{\Omega_l} \left( | p^{-1} |\chi| \leq (1-\eps) |A| \wedge p \geq q \right)\\
  & + &
    \prob_{\Omega_l} \left( | p^{-1} |\chi| \geq (1+\eps) |A| \wedge p \geq q \right) +
    \prob_{\Omega_l} \left( p < q \right) \\
    & \underset{Le.~\ref{le:upper_tail}-\ref{le:lower_tail}}{\leq} &
    \exp\left( - \frac{n}{8} \eps^2 \right)  + \exp\left( - \frac{n}{12} \eps^2 \right) +
    l \exp\left( - \frac{n}{12} \right) \\
    & \leq & \frac{\delta}{3} + \frac{\delta}{3} + \frac{\delta}{3}
\end{eqnarray*}
\end{proof}
\subsection{Unbiasedness\label{sec:unbiasedness}}
Let $M$ be large enough such that $p^{-1} \leq M$ a.s. (e.g., we can choose $M = f^{-l}$).
Then we can derive from Lemma~\ref{le:prob_invariant_simple} using $h(x) = x$ and $h(x) = M+1-x$ that
for all $s \in A$:
\begin{eqnarray*}
  \expect_{\Omega_l} [ p^{-1} I(s \in \chi) ] & = &
    \expect_{\Omega_l} [ I(p \geq M^{-1}) p^{-1} I(s \in \chi) ] \leq 1 \\
  \expect_{\Omega_l} [ M + 1 - p^{-1} I(s \in \chi) ] & = &
    \expect_{\Omega_l} [ I(p \geq M^{-1}) (M + 1 - p^{-1} I(s \in \chi)) ] \leq M
\end{eqnarray*}
which implies $\expect_{\Omega_l} [ p^{-1} I(s \in \chi) ] = 1$.
By linearity of expectation we conclude
\[
  \expect_{\Omega_l} [ p^{-1} |\chi| ] =
    \sum_{s \in A} \expect_{\Omega_l} [ p^{-1} I(s \in \chi) ] = |A| \textrm{.}
\]


\end{document}