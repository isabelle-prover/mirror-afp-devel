\documentclass[11pt,a4paper]{article}
\usepackage[T1]{fontenc}
\usepackage{isabelle,isabellesym}

\usepackage{url}
\usepackage{amssymb}
\usepackage{xspace}

% this should be the last package used
\usepackage{pdfsetup}

% urls in roman style, theory text in math-similar italics
\urlstyle{rm}
\isabellestyle{it}

\newcommand\isafor{\textsf{Isa\kern-0.15exF\kern-0.15exo\kern-0.15exR}}
\newcommand\ceta{\textsf{C\kern-0.15exe\kern-0.45exT\kern-0.45exA}}
\newcommand\sms{\mathrel{\succ_{\mathit{ms}}}}

\begin{document}

\title{The Generalized Multiset Ordering is NP-Complete}
\author{Ren\'e Thiemann \and Lukas Schmidinger}
\maketitle

\begin{abstract}
We consider the problem of comparing two multisets via the
generalized multiset ordering. We show that the corresponding decision problem
is NP-complete. To be more precise, we encode
multiset-comparisons into propositional formulas or into conjunctive normal forms 
of quadratic size;
we further prove that satisfiability of conjunctive normal forms can be encoded as
multiset-comparison problems of linear size. 

As a corollary, we also show that the problem of deciding whether
two terms are related by a recursive path order is NP-hard,
provided the recursive path order is based on the generalized multiset ordering.
\end{abstract}

\tableofcontents

\section{Introduction}

Given a transitive and irreflexive relation $\succ$ on elements, 
it can be extended to a relation on multisets (the \emph{multiset ordering} $\sms$)
where for two multisets $M$ and $N$ the relation $M \sms N$ is defined in a way that $N$ is obtained from $M$ by
replacing some elements $a \in M$ by arbitrarily many elements $b_1,\dots,b_n$  which are
all smaller than $a$: $a \succ b_i$ for all $1 \leq i \leq n$.

Now, given $\succ$, $M$, and $N$, it is easy to decide $M \sms N$: it is equivalent
to demand $M \neq N$ and for each $b \in N \setminus M$ there must be some $a \in M \setminus N$ 
such that $a \succ b$.

The \emph{generalized multiset ordering} is defined in terms of two 
relations $\succ$ and $\succsim$. Here, one may additionally replace each element $a \in M$ by exactly one
element $b$ that satisfies $a \succsim b$. 
The multiset ordering is an instance of the generalized multiset ordering by choosing ${\succsim}$ as the
equality relation ${=}$.

The generalized multiset ordering is used in some definitions of  
the recursive path order (the original RPO \cite{RPO} is defined via the multiset ordering, the variants of RPO \cite{RPO_NP,RPO_NPC} use the generalized multiset ordering instead)  
so that more terms are in relation. A downside of the generalization is that the decision problem of whether
two multisets are in relation becomes NP-complete, and also the decision
problem for the RPO-variant in \cite{RPO_NPC} is NP-complete.

In this AFP-entry we formalize NP-completeness of the generalized multiset ordering: 
we provide an ${\cal O}(n^2)$ encoding of multiset-comparisons into propositional formulas (using connectives
${\vee},{\wedge},{\neg},{\rightarrow},{\leftrightarrow}$), 
an ${\cal O}(n^2)$ encoding of multiset-comparisons into conjunctive normal forms (CNF),
and an ${\cal O}(n)$ encoding of CNFs into multiset-comparisons.
Moreover, we verify an ${\cal O}(n^2)$ encoding from a CNF into an RPO-constraint.

Our formalization is based on proofs in \cite{RPO_NP} (in NP) and \cite{RPO_NPC} (NP-hardness).

\input{session}



\bibliographystyle{abbrv}
\bibliography{root}

\end{document}

