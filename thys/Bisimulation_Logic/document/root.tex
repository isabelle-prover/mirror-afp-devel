\documentclass[11pt,a4paper]{article}
\usepackage[T1]{fontenc}
\usepackage{isabelle,isabellesym}
\usepackage{amssymb}


% this should be the last package used
\usepackage{pdfsetup}

% urls in roman style, theory text in math-similar italics
\urlstyle{rm}
\isabellestyle{it}


\begin{document}

\title{A meta-modal logic for bisimulations}
\author{Alfredo Burrieza \and Fernando Soler-Toscano \and Antonio Yuste-Ginel}

\maketitle

\begin{abstract}
  Bisimulations are a fundamental formal tool in the model theory of
  standard modal logic. Roughly speaking, bisimulations provide a
  clear answer to a foundational model-theoretical question: Given two
  (Kripke-style) models, what conditions are sufficient and necessary
  for them to satisfy the same modal formulas?  We propose a modal
  study of the notion of bisimulation. We extend the basic modal
  language with a new modality $[b]$, whose intended meaning is
  universal quantification over all states that are bisimilar to the
  current one. We provide a sound and complete axiomatisation of the
  class of all pairs of Kripke models linked by bisimulations.
\end{abstract}

\tableofcontents

% sane default for proof documents
\parindent 0pt\parskip 0.5ex

{\medskip}

% generated text of all theories
\input{session}

\bibliographystyle{abbrv}
\bibliography{root}

\end{document}
