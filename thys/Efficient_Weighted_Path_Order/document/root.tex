\documentclass[11pt,a4paper]{article}
\usepackage[T1]{fontenc}
\usepackage{isabelle,isabellesym}

\usepackage{url}
\usepackage{amssymb}
\usepackage{xspace}

\usepackage{authblk}

% this should be the last package used
\usepackage{pdfsetup}

% urls in roman style, theory text in math-similar italics
\urlstyle{rm}
\isabellestyle{it}

\newcommand\isafor{\textsf{Isa\kern-0.15exF\kern-0.15exo\kern-0.15exR}}
\newcommand\ceta{\textsf{C\kern-0.15exe\kern-0.45exT\kern-0.45exA}}

\begin{document}

\title{A Verified Efficient Implementation of the Weighted Path Order\footnote{This research was supported by the Austrian Science Fund (FWF) project I~5943.}}
\author{Ren\'e Thiemann}
\author{Elias Wenninger}
\affil{University of Innsbruck}
\maketitle

\begin{abstract}
The Weighted Path Order (WPO) of Yamada is a powerful technique for proving
termination \cite{WPO_form_paper,WPO-PPDP,WPO}. In a previous AFP entry \cite{WPO_AFP}, the WPO was defined and properties
of WPO have been formally verified. However, the implementation of WPO was naive, leading 
to an exponential runtime in the worst case.

Therefore, in this AFP entry we provide a poly-time implementation of WPO.
The implementation is based on memoization. Since WPO generalizes the recursive path order (RPO) \cite{RPO},
we also easily derive an efficient implementation of RPO.
\end{abstract}

\tableofcontents


\input{session}



\bibliographystyle{abbrv}
\bibliography{root}

\end{document}

