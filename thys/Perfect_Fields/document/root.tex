\documentclass[11pt,a4paper]{article}
\usepackage[T1]{fontenc}
\usepackage{isabelle,isabellesym}
\usepackage{amsfonts, amsmath, amssymb}

% this should be the last package used
\usepackage{pdfsetup}
\usepackage[shortcuts]{extdash}

% urls in roman style, theory text in math-similar italics
\urlstyle{rm}
\isabellestyle{rm}


\begin{document}

\title{Perfect Fields}
\author{Manuel Eberl, Katharina Kreuzer}
\maketitle

\begin{abstract}
This entry provides a type class for \emph{perfect fields}. 
A perfect field $K$ can be characterized by one of the following equivalent conditions \cite{wiki:perfect_field}:
\begin{enumerate}
\item Any irreducible polynomial $p$ is separable, i.e.\ $\gcd(p,p') = 1$, or, equivalently, $p' \neq 0$.
\item Either $\text{char}(K) = 0$ or $\text{char}(K) = p > 0$ and the Frobenius endomorphism $x \mapsto x^p$ is surjective (i.e.\ every element of $K$ has a $p$-th root).
\end{enumerate}
We define perfect fields using the second characterization and show the equivalence to the first characterization.
The implication ``$2 \Rightarrow 1$'' is relatively straightforward using the injectivity of the Frobenius homomorphism.\\

\noindent Examples for perfect fields are \cite{wiki:perfect_field}:
\begin{itemize}
\item any field of characteristic $0$ (e.g.\ $\mathbb{R}$ and $\mathbb{C}$)
\item any finite field (i.e.\ $\mathbb{F}_q$ for $q=p^n$, $n > 0$ and $p$ prime)
\item any algebraically closed field (for example the formal Puiseux series over finite fields)
\end{itemize}
\end{abstract}


\newpage
\tableofcontents

\newpage
\parindent 0pt\parskip 0.5ex

\input{session}

\raggedright
\bibliographystyle{abbrv}
\bibliography{root}

\end{document}

