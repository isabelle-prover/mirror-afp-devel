\documentclass[10pt,a4paper]{article}
\usepackage[T1]{fontenc}
\usepackage{isabelle,isabellesym}
\usepackage{a4wide}
\usepackage[english]{babel}

% further packages required for unusual symbols (see also
% isabellesym.sty), use only when needed

\usepackage{amssymb}
  %for \<leadsto>, \<box>, \<diamond>, \<sqsupset>, \<mho>, \<Join>,
  %\<lhd>, \<lesssim>, \<greatersim>, \<lessapprox>, \<greaterapprox>,
  %\<triangleq>, \<yen>, \<lozenge>
\usepackage{wasysym}
% for \<hole>

% this should be the last package used
\usepackage{pdfsetup}

% urls in roman style, theory text in math-similar italics
\urlstyle{rm}
\isabellestyle{literal}

% for uniform font size
\renewcommand{\isastyle}{\isastyleminor}

\renewcommand{\isamarkupsection}[1]{\section{#1}}
\renewcommand{\isamarkupsubsection}[1]{\subsection{#1}}
\renewcommand{\isamarkupsubsubsection}[1]{\paragraph{#1}}

\newcommand{\munta}{\textsc{Munta}}
\newcommand{\muntac}{\textsc{Muntac}}
\newcommand{\mlunta}{\textsc{MLunta}}

\begin{document}

\title{\muntac: A Verified Certificate Checker for Timed Automata}
\author{Simon Wimmer}
\date{}

\maketitle
\begin{abstract}
  \munta\ is a verified model checker for timed automata.
  This entry presents \muntac, an extension of \munta\ that can check certificates for
  B\"uechi and reachability properties generated by other model checkers.
  This work has been described in detail in a PhD thesis \cite{wimmer-phd}
  and conference publications~\cite{muntac-tacas,muntac-formats}.
  This entry supersedes earlier versions of \muntac\ hosted on GitHub.
\end{abstract}

\setcounter{tocdepth}{2}
\tableofcontents
\newpage

\section*{Introduction}

\munta\ is a verified model checker for timed automata.
This entry presents \muntac, an extension of \munta\ that can check certificates for
B\"uechi and reachability properties generated by other model checkers.
This work has been has been described in detail in a PhD thesis \cite{wimmer-phd}
and presented in peer-reviewed conference publications~\cite{muntac-tacas,muntac-formats}.
This entry supersedes earlier versions of \muntac\ hosted on GitHub:
\url{https://github.com/wimmers/munta}.

\paragraph{Usage}
Theory \texttt{Simple\_Network\_Language\_Certificate\_Code} (page \pageref{export-muntac}) describes
how to obtain executable code for and run \muntac\ from within Isabelle.

This entry also provides a build setup for the \textsc{muntac} executable
using the MLton and the Poly/ML compilers for Standard ML.
Instructions to build and run executables are given in theory \texttt{Munta\_Certificate\_Compile\_MLton}
(on page \pageref{build-mlton}) and \texttt{Munta\_Certificate\_Compile\_Poly}
(on page \pageref{build-poly}), respectively.
The MLton executable has much better performance on a single core, while the Poly/ML executable
offers the ability for parallel certificate checking.
A number of example model files can be found in the \texttt{benchmarks} directory.

This entry also bundles up the unverified model checker
\mlunta\footnote{Git commit \texttt{99a3b55ca7f56ebe0bf0e5f8384adf73fa074bad}},
which has previously been used to generate test certificates for \muntac.
The aforementioned build setups also exercise this tool chain by first compiling \mlunta\ using
MLton and then generating certificates for some of the benchmarks,
which are in turn checked by \muntac.

\munta\ itself can also produce reachability certificates for \muntac, as
demonstrated on page~\pageref{self-check}.
B\"uchi certificates from the
artifact\footnote{\url{https://doi.org/10.6084/m9.figshare.12620582.v1}}
associated with the FORMATS paper~\cite{muntac-formats} remain compatible with \muntac.
Note that the \texttt{-i 4} option must be specified explicitly, for example:
\begin{verbatim}
muntac -i 4 -m cc4/cc4.muntax -r cc4/cc4.renaming -c cc4/cc4-tchecker-ndfs13.cert
\end{verbatim}

\newpage

% sane default for proof documents
\parindent 0pt\parskip 0.5ex

% generated text of all theories
\input{session}

% optional bibliography
\bibliographystyle{abbrv}
\bibliography{root}

\end{document}

%%% Local Variables:
%%% mode: latex
%%% TeX-master: t
%%% End:

