\documentclass[11pt,a4paper]{article}
\usepackage[utf8x]{inputenc}
\usepackage[LGR,T1]{fontenc}
\usepackage{isabelle,isabellesym}
\usepackage{amsfonts, amsmath, amssymb}
\usepackage[english]{babel}
\newcommand{\oeiscite}[1]{\cite[\href{https://oeis.org/#1}{#1}]{oeis}}

% this should be the last package used
\usepackage{pdfsetup}
\usepackage[shortcuts]{extdash}

% urls in roman style, theory text in math-similar italics
\urlstyle{rm}
\isabellestyle{rm}


\begin{document}

\title{The Partition Function and the Pentagonal Number Theorem}
\author{Manuel Eberl}
\maketitle

\begin{abstract}
The partition function $p(n)$~\oeiscite{A000041} gives the number of ways to write a non-negative
integer $n$ as a sum of positive integers, without taking order into account.

This entry uses the Jacobi Triple Product (already available in the AFP) to give a short proof of 
the \emph{Pentagonal Number Theorem}, which is the statement that the generating function 
$F(X) = \sum_{n\geq 0} p(n) X^n$ of the partition function satisfies:
\[F(X)^{-1} = \sum_{k\in\mathbb{Z}} (-1)^k X^{k(3k-1)/2} = 1 - x - x^2 + x^5 + x^7 - x^{12} - x^{15} + \ldots\]
The numbers $g_k = \frac{1}{2}k(3k-1)$ appearing in the exponents are the
\emph{generalised pentagonal numbers}~\oeiscite{A001318}.

As further corollaries of this, an upper bound for $p(n)$ and the recurrence relation
\[p(n) = \sum_{\substack{k\in\mathbb{Z}\setminus\{0\} \\ g_k \leq n}} (-1)^{k+1} p(n-g_k)\]
are proved. The latter also yields an algorithm to compute the numbers
$p(0), \ldots, p(n)$ simultaneously in time roughly $n^{2 + o(1)}$. This algorithm is
implemented and proved correct at the end of this entry using the Imperative-HOL Refinement Framework.
\end{abstract}
\newpage

\tableofcontents

\newpage
\parindent 0pt\parskip 0.5ex

\input{session}

\raggedright

\bibliographystyle{abbrv}
\bibliography{root}

\end{document}

