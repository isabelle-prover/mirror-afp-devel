\documentclass[11pt,a4paper]{article}
\usepackage[T1]{fontenc}
\usepackage{isabelle,isabellesym}

% this should be the last package used
\usepackage{pdfsetup}

% urls in roman style, theory text in math-similar italics
\urlstyle{rm}
\isabellestyle{it}


\begin{document}

\title{Formalized Burrows-Wheeler Transform}
\author{Louis Cheung and Christine Rizkallah}
\maketitle

\begin{abstract}
The Burrows-Wheeler transform (BWT)~\cite{Burrows_Tech_1994} is an invertible lossless
transformation that permutes input sequences into alternate sequences
of the same length that frequently contain long localized regions
that involve clusters consisting of just a few distinct symbols, and
sometimes also include long runs of same-symbol repetitions.
Moreover, there is a one-to-one correspondence between the BWT and suffix arrays~\cite{Manber_SIAM_1993}.
As a consequence, the BWT is widely used in data compression
and as an indexing data structure for pattern search.
In this formalization~\cite{Cheung_Zenodo_2024}, we present the formal verification of both the
BWT and its inverse,
building on a formalization of suffix arrays~\cite{Cheung_AFP_2024}.
This is the artefact of our CPP paper~\cite{Cheung_CPP_2025}.
\end{abstract}

\tableofcontents

% include generated text of all theories
\input{session}

\bibliographystyle{abbrv}
\bibliography{root}

\end{document}
