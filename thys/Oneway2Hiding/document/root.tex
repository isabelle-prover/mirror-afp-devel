\documentclass[11pt,a4paper]{article}
\usepackage[T1]{fontenc}
\usepackage{isabelle,isabellesym}
\usepackage{geometry}
\usepackage{amsmath, amssymb}
\usepackage{wasysym}
\usepackage{stmaryrd}

% this should be the last package used
\usepackage{pdfsetup}

% urls in roman style, theory text in math-similar italics
\urlstyle{rm}
\isabellestyle{it}

% for uniform font size
%\renewcommand{\isastyle}{\isastyleminor}


\begin{document}

\title{The Oneway to Hiding Theorem\thanks{Supported by the research training group ConVeY of the German Research Foundation under grant GRK 2428 and the Deutsche Forschungsgemeinschaft (DFG, German Research Foundation) -- NI 491/18-1, by the ERC consolidator grant CerQuS (Certified Quantum Security,  819317), by
the Estonian Centre of Excellence in IT (EXCITE, TK148), by
the Estonian Centre of Excellence "Foundations of the Universe" (TK202), and by
the Estonian Research Council PRG grant "Secure Quantum Technology" (PRG946).}}

\author{Katharina Heidler\and Dominique Unruh}
\maketitle

\begin{abstract}
As the standardization process for post-quantum cryptography progresses, the need for computer-verified security proofs against classical and quantum attackers increases. 
Even though some tools are already tackling this issue, none are foundational. 
We take a first step in this direction and present a complete formalization of the One-way to Hiding (O2H) Theorem, a central theorem for security proofs against quantum attackers. 
With this new formalization, we build more secure foundations for proof-checking tools in the quantum setting.
Using the theorem prover Isabelle, we verify the semi-classical O2H Theorem by Ambainis, Hamburg and Unruh (Crypto 2019) in different variations. We also give a novel (and for the formalization simpler) proof to the O2H Theorem for mixed states and extend the theorem to non-terminating adversaries.
This work provides a theoretical and foundational background for several verification tools and for security proofs in the quantum setting.
\end{abstract}

A paper describing this work in more detail appeared at \cite{HeidlerU25}.

\nocite{Ambainis19}

\tableofcontents

% sane default for proof documents
\parindent 0pt\parskip 0.5ex

% generated text of all theories
\input{session}

% optional bibliography
\bibliographystyle{abbrv}
\bibliography{root}

\end{document}

%%% Local Variables:
%%% mode: latex
%%% TeX-master: t
%%% End:
