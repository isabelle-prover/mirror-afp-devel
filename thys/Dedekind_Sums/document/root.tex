\documentclass[11pt,a4paper]{article}
\usepackage[utf8x]{inputenc}
\usepackage[LGR,T1]{fontenc}
\usepackage{isabelle,isabellesym}
\usepackage{amsfonts, amsmath, amssymb}

% this should be the last package used
\usepackage{pdfsetup}
\usepackage[shortcuts]{extdash}

% urls in roman style, theory text in math-similar italics
\urlstyle{rm}
\isabellestyle{rm}


\begin{document}

\title{Dedekind Sums}
\author{Manuel Eberl, Anthony Bordg, Lawrence C.\ Paulson, Wenda Li}
\maketitle

\begin{abstract}
For integers $h$, $k$, the Dedekind sum is defined as
\[s(h,k) = \sum_{r=1}^{k-1} \frac{r}{k} \left(\left\{\frac{hr}{k}\right\} - \frac{1}{2}\right) \]
where $\{x\} = x - \lfloor x\rfloor$ denotes the fractional part of $x$.

These sums occur in various contexts in analytic number theory, e.g.\ in the functional
equation of the Dedekind $\eta$ function or in the study of modular forms.

We give the definition of $s(h,k)$ and prove its basic properties, including the reciprocity law
\[s(h,k) + s(k,h) = \frac{1}{12hk} + \frac{h}{12k} + \frac{k}{12h} - \frac{1}{4} \]
and various congruence results.

Our formalisation follows Chapter~3 of Apostol's
\emph{Modular Functions and Dirichlet Series in Number Theory}~\cite{apostol} and contains
all facts related to Dedekind sums from it (without the exercises).
\end{abstract}

\tableofcontents

\newpage
\parindent 0pt\parskip 0.5ex

\input{session}

\raggedright
\nocite{apostol}

\bibliographystyle{abbrv}
\bibliography{root}

\end{document}

