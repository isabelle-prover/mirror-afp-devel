\documentclass[11pt,a4paper]{article}
\usepackage[T1]{fontenc}
\usepackage{isabelle,isabellesym}
\usepackage{amsfonts,amsmath, amssymb}

\usepackage[only,bigsqcap]{stmaryrd}

% this should be the last package used
\usepackage{pdfsetup}

% urls in roman style, theory text in math-similar italics
\urlstyle{rm}
\isabellestyle{it}

\begin{document}

\title{Negatively Associated Random Variables}
\author{Emin Karayel}
\maketitle

\begin{abstract}
Negative Association is a generalization of independence for random variables, that retains some
of the key properties of independent random variables. In particular closure properties, such as
composition with monotone functions, as well as, the well-known Chernoff-Hoeffding bounds.

This entry introduces the concept and verifies the most important closure properties, as well as,
the concentration inequalities. It also verifies the FKG inequality, which is a generalization of
Chebyshev's sum inequality for distributive lattices and a key tool for establishing negative
association, but has also many applications beyond the context of negative association, in 
particular, statistical physics and graph theory.

As an example, permutation distributions are shown to be negatively associated, from which
many more sets of negatively random variables can be derived, such as, e.g., n-subsets, or the 
the balls-into-bins process.

Finally, the entry derives a correct false-positive rate for Bloom filters using the library.
\end{abstract}

\tableofcontents

\parindent 0pt\parskip 0.5ex

\input{session}

\bibliographystyle{abbrv}
\bibliography{root}

\end{document}
