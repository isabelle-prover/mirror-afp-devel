\documentclass[11pt,a4paper]{article}
\usepackage[T1]{fontenc}
\usepackage{isabelle,isabellesym}
\usepackage{amsfonts, amsmath, amssymb}

% this should be the last package used
\usepackage{pdfsetup}

% urls in roman style, theory text in math-similar italics
\urlstyle{rm}
\isabellestyle{it}


\begin{document}

\title{Continued Fractions}
\author{Manuel Eberl}
\maketitle

\begin{abstract}
This article provides a formalisation of continued fractions of real numbers and their basic properties.
It also contains a proof of the classic result that the irrational numbers with periodic continued
 fraction expansions are precisely the quadratic irrationals, i.\,e.\ real numbers that fulfil a non-trivial quadratic equation $a x^2 + b x + c = 0$ with integer coefficients.

Particular attention is given to the continued fraction expansion of $\sqrt{D}$ for a non-square
 natural number $D$. Basic results about the length and structure of its period are provided,
 along with an executable algorithm to compute the period (and from it, the entire expansion).

This is then also used to provide a fairly efficient, executable, and fully formalised algorithm to
compute solutions to Pell's equation $x^2 - D y^2 = 1$. The performance is sufficiently good to
find the solution to Archimedes's cattle problem in less than a second on a typical computer.
This involves the value $D = 410286423278424$, for which the solution has over 200000 decimals.

Lastly, a derivation of the continued fraction expansions of Euler's number $e$ and an executable
function to compute continued fraction expansions using interval arithmetic is also provided.
\end{abstract}

\newpage
\tableofcontents
\newpage
\parindent 0pt\parskip 0.5ex

\input{session}

\nocite{khinchin}

\bibliographystyle{abbrv}
\bibliography{root}

\end{document}

%%% Local Variables:
%%% mode: latex
%%% TeX-master: t
%%% End:
