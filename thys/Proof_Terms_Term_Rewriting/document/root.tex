\documentclass[11pt,a4paper]{article}
\usepackage[T1]{fontenc}
\usepackage{isabelle,isabellesym}

\usepackage{amssymb}
\usepackage{authblk}

%\usepackage{eurosym}

%\usepackage[only,bigsqcap,bigparallel,fatsemi,interleave,sslash]{stmaryrd}

%\usepackage{eufrak}

%\usepackage{textcomp}

% this should be the last package used
\usepackage{pdfsetup}

% urls in roman style, theory text in math-similar italics
\urlstyle{rm}
\isabellestyle{it}

\newcommand\isafor{\textsf{Isa\kern-0.15exF\kern-0.15exo\kern-0.15exR}}
\newcommand\ceta{\textsf{C\kern-0.15exe\kern-0.45exT\kern-0.45exA}}

% for uniform font size
%\renewcommand{\isastyle}{\isastyleminor}


\begin{document}

\title{Proof Terms for Term Rewriting}
\author{Christina Kirk (Kohl)}
\affil{University of Innsbruck, Austria}
\maketitle

\begin{abstract}
  Proof terms are first-order terms that represent reductions in term rewriting.
  They were initially introduced in \cite{vOdV02} and \cite[Chapter 8]{TeReSe}
  by van Oostrom and de Vrijer to study equivalences of reductions in left-linear
  rewrite systems.
  This entry formalizes proof terms for multi-steps in first-order term rewrite systems.
  We define simple proof terms (i.e., without a composition operator) and establish
  the correspondence to multi-steps: each proof term represents a multi-step
  with the same source and target, and every multi-step can be expressed as a proof term.
  The formalization moreover includes operations on proof terms, such as residuals,
  join, and deletion and a method for labeling proof term sources to identify
  overlaps between two proof terms.

  This formalization is part of the \emph{Isabelle Formalization of Rewriting}
  \href{http://cl-informatik.uibk.ac.at/isafor/}{\isafor{}} and is an essential component
  of several formalized confluence and commutation results
  involving multi-steps~\cite{KM23a, KM23b, KM23c, KM25}.
\end{abstract}

\tableofcontents

% sane default for proof documents
\parindent 0pt\parskip 0.5ex

% generated text of all theories
\input{session}

% optional bibliography
\bibliographystyle{abbrv}
\bibliography{root}

\end{document}

%%% Local Variables:
%%% mode: latex
%%% TeX-master: t
%%% End:
