\documentclass[11pt,a4paper]{article}
\usepackage{isabelle,isabellesym}
\usepackage{graphicx}
\graphicspath {{figures/}}
\usepackage[T1]{fontenc}


% necessary to have the box symbol, used by CSP
\usepackage{authblk} 
\usepackage{latexsym}
\usepackage{amssymb}
\usepackage{amsmath}

\usepackage{pdfsetup}

% urls in roman style, theory text in math-similar italics
\urlstyle{rm}
\isabellestyle{it}

% for uniform font size
%\renewcommand{\isastyle}{\isastyleminor}
\pagestyle{plain}

\begin{document}
\title{Morley's Theorem\thanks{This work has been supported by the French government under the "France 2030" program, as part of the SystemX Technological Research Institute within the CVH project.}}
\author{Benjamin Puyobro}
\affil{Université Paris Saclay, IRT SystemX, LMF, CNRS}

\maketitle
\tableofcontents
\newpage
% sane default for proof documents
\parindent 0pt\parskip 0.5ex
% generated text of all theories
\input{session}
% Bibliographie
\begin{thebibliography}{99} % L'argument "99" est pour l'espace réservé au numéro max.
\bibitem{Triangle-AFP}
Manuel Eberl
\textit {Basic Geometric Properties of Triangles}, Archive of Formal Proofs, December 2015
\url{https://isa-afp.org/entries/Triangle.html}
\bibitem{source1}
Morley theorem
\url{http://denisevellachemla.eu/Alain-Connes-Theoreme-Morley.pdf}.
\bibitem{source3}
Wiedijk's catalogue "Formalizing 100 Theorems"
\url{https://www.cs.ru.nl/~freek/100/, It appears at position 121 ...}.
\end{thebibliography}
% optional bibliography
%\bibliographystyle{abbrv}
%\bibliography{references}
\end{document}

%%% Local Variables:
%%% mode: latex
%%% TeX-master: t
%%% End:
