\documentclass[11pt,a4paper]{article}
\usepackage[T1]{fontenc}
\usepackage{isabelle,isabellesym}

\usepackage{amsmath}
\usepackage{amssymb}
\usepackage[utf8]{inputenc}

% this should be the last package used
\usepackage{url}
\usepackage{pdfsetup}

% urls in roman style, theory text in math-similar italics
\urlstyle{rm}
\isabellestyle{it}

% for uniform font size
%\renewcommand{\isastyle}{\isastyleminor}


\begin{document}

\title{Clique is not solvable by monotone circuits of polynomial size\footnote{We thank Lev Gordeev for several clarification regarding his proof, 
for his explanation of the history of the underlying proof idea, 
and for a lively and ongoing interesting discussion on
how his draft can be repaired.}}
\author{Ren\'e Thiemann\\
  {\small University of Innsbruck}}
\maketitle

\begin{abstract}
Given a graph $G$ with $n$ vertices and a number $s$, the decision problem Clique
asks whether $G$ contains a fully connected subgraph with $s$ vertices.
For this NP-complete problem there exists
a non-trivial lower bound: 
no monotone circuit of a size that is polynomial in $n$ can solve Clique.

This entry provides an Isabelle/HOL formalization of a concrete 
lower bound (the bound is $\sqrt[7]{n}^{\sqrt[8]{n}}$ for the fixed choice 
of $s = \sqrt[4]{n}$), 
following a proof by Gordeev.
\end{abstract}

\tableofcontents

% sane default for proof documents
\parindent 0pt\parskip 0.5ex

\section{Introduction}
In this AFP submission we verify the result, that no polynomial-sized circuit 
can implement the Clique problem.

We arrived at this formalization by trying to verify 
an unpublished draft of Gordeev \cite{PNP}, which tries to show
that Clique cannot be solved by any polynomial-sized circuit,
including non-monotone ones, where the concrete exponential
lower bound is $\sqrt[7]{n}^{\sqrt[8]{n}}$ for graphs 
with $n$ vertices and cliques of size $s = \sqrt[4]n$.

Although there are some flaws in that draft, all of these disappear
if one restricts to monotone circuits. Consequently, the claimed
lower bound is valid for monotone circuits.

We verify a simplified version of Gordeev's proof, where those parts that
deal with negations in circuits have been eliminated from definitions and proofs.

Gordeev's work itself was inspired by ``Razborov's theorem'' 
in a textbook by Papadimitriou
\cite{P94}, which states that Clique cannot be encoded with a monotone
circuit of polynomial size. 
However the proof in the draft uses a construction based on 
the sunflower lemma of Erdős and Rado \cite{erdos_rado},
following a proof in Boppana and 
Sipser \cite{BS90}. 
There are further proofs on lower bounds of monotone circuits for Clique.
For instance, an early result is due to Alon and Boppana \cite{AlonB87},
where they show a slightly different lower bound (using a differently structured
proof without the construction based on sunflowers.)


\input{session}

\bibliographystyle{abbrv}
\bibliography{root}

\end{document}

%%% Local Variables:
%%% mode: latex
%%% TeX-master: t
%%% End:
