\documentclass[11pt,notitlepage,a4paper]{report}
\usepackage[T1]{fontenc}
\usepackage{isabelle,isabellesym,eufrak}
\usepackage{amssymb,amsmath}
\usepackage[english]{babel}

% XYPic package, for drawing commutative diagrams.
\input{xy}
\xyoption{curve}
\xyoption{arrow}
\xyoption{matrix}
\xyoption{2cell}
\xyoption{line}
\UseAllTwocells

% this should be the last package used
\usepackage{pdfsetup}

% urls in roman style, theory text in math-similar italics
\urlstyle{rm}
\isabellestyle{it}

\newcommand\labarrow[1]{{\stackrel{#1}{\longrightarrow}}}

\begin{document}

\title{Enriched Category Basics}
\author{Eugene W. Stark\\[\medskipamount]
        Department of Computer Science\\
        Stony Brook University\\
        Stony Brook, New York 11794 USA}
\maketitle

\begin{abstract}
The notion of an enriched category generalizes the concept of category by replacing the hom-sets
of an ordinary category by objects of an arbitrary monoidal category.
In this article we give a formal definition of enriched categories and we give formal proofs
of a relatively narrow selection of facts about them.
One of the main results is a proof that a closed monoidal category can be regarded as a
category ``enriched in itself''.
The other main result is a proof of a version of the Yoneda Lemma for enriched categories.
\end{abstract}

\newpage

\phantomsection
\addcontentsline{toc}{chapter}{Contents}
\tableofcontents

\phantomsection
\addcontentsline{toc}{chapter}{Introduction}
\chapter*{Introduction}

The notion of an enriched category \cite{kelly-enriched-category}
generalizes the concept of category by replacing the hom-sets of an ordinary category by
objects of an arbitrary monoidal category ${\cal V}$.
The choice, for each object $a$, of a distinguished element $id~a : a \rightarrow a$
as an identity, is replaced by an arrow $Id~a : {\cal I} \rightarrow Hom~a~a$ of ${\cal V}$.
The composition operation is similarly replaced by a family of arrows
$Comp~a~b~c : Hom~B~C \otimes Hom~A~B \rightarrow Hom~A~C$ of ${\cal V}$.
The identity and composition are required to satisfy unit and associativity laws which are
expressed as commutative diagrams in ${\cal V}$.
Of particular interest is the case in which ${\cal V}$ is symmetric monoidal and closed;
in that case, as Kelly states (\cite{kelly-enriched-category}, Section 1.6):
 ``The structure of ${\cal V}$-{\bf CAT} then becomes rich enough to permit of Yoneda-lemma
arguments formally identical with those in {\bf CAT}.''

The goal of this article is to formalize the basic definition of enriched category and
some related notions, and to prove a relatively narrow selection of facts about these definitions.
For reference and inspiration, we follow the early sections of the book by
Kelly \cite{kelly-enriched-category}; however a comprehensive formalization of the material
in that book is explicitly not our objective here.
Rather, beyond the basic definitions we are primarily interested in the following two results:
(1) that a closed monoidal category can be regarded as a category ``enriched in itself''; and
(2) the Yoneda Lemma for enriched categories
(specifically, the weak form considered in Section 1.9 of \cite{kelly-enriched-category}).
We needed the basic definitions and result (1) for use in a separate article
\cite{ResiduatedTransitionSystem2-AFP}.
Although this material could have been included as part of that other article,
as it is general material that does not depend on the specific application considered there,
it seemed best to present it as a stand-alone development that would be more readily
accessible for use by others.
As far as result (2) is concerned, we originally formalized and proved it as part of our exploration
leading up to \cite{ResiduatedTransitionSystem2-AFP}.
Ultimately, we did not find result (2) to be necessary for the satisfactory development
of that work, but as it is a result of general interest whose formalization did involve
some struggle to achieve, it seems worthwhile to include it here.

This article is organized as follows:
In Chapter \ref{closed_monoidal_categories_chapter} we give formal definitions for the
notions ``closed monoidal category'' and ``cartesian closed monoidal category'' and
prove some facts about them.  This builds on the formal development of the theory of
monoidal categories in our previous article \cite{MonoidalCategory-AFP}.
The main goals of this section are to prove some general facts about exponentials that are
used in \cite{ResiduatedTransitionSystem2-AFP}, and to do most of the preliminary work
(the parts that do not specifically depend on the definition of enriched category)
involved in showing that a closed monoidal category is ``enriched in itself''.
In Chapter \ref{enriched_categories_chapter} we give definitions for ``enriched category''
and the related notions ``enriched functor,'' ``enriched natural transformation,''
and ``underlying category,'' and we complete the formal statement and proof of
``self-enrichment.''
We then continue with the definition of the opposite of an enriched category,
give definitions for the notions of covariant and contravariant enriched hom functors,
and prove corresponding covariant and contravariant versions of the Yoneda Lemma.

\chapter{Closed Monoidal Categories}
\label{closed_monoidal_categories_chapter}

  \input{ClosedMonoidalCategory.tex}
  \input{CartesianClosedMonoidalCategory.tex}

\chapter{Enriched Categories}
\label{enriched_categories_chapter}

  \input{EnrichedCategory.tex}

\clearpage
\phantomsection
\addcontentsline{toc}{chapter}{Bibliography}

\bibliographystyle{abbrv}
\bibliography{root}

\end{document}
