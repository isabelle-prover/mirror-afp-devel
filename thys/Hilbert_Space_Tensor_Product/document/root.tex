\documentclass[11pt,a4paper]{article}
\usepackage[T1]{fontenc}
\usepackage{isabelle,isabellesym}
\usepackage{geometry}
\usepackage{amsmath, amssymb}
\usepackage{wasysym}
\usepackage{stmaryrd}

% this should be the last package used
\usepackage{pdfsetup}

% urls in roman style, theory text in math-similar italics
\urlstyle{rm}
\isabellestyle{it}

% for uniform font size
%\renewcommand{\isastyle}{\isastyleminor}


\begin{document}

\title{Tensor Products in Hilbert Spaces\thanks{Supported by the ERC consolidator grant CerQuS (819317), the PRG team grant “Secure Quantum Technology” (PRG946) from the Estonian Research Council, the Estonian Centre of Exellence in IT (EXCITE) funded by ERDF, and the Estonian Cluster of Excellence ``Foundations of the Universe'' (TK202).}}
\author{Dominique Unruh}
\maketitle

\begin{abstract}
  We formalize the tensor product of Hilbert spaces, and related material.
  Specifically, we define the product of vectors in Hilbert spaces, of operators on Hilbert spaces, and of subspaces of Hilbert spaces, and of von Neumann algebras, and study their properties.

  The theory is based on the AFP entry \texttt{Complex\_Bounded\_Operators} that introduces Hilbert spaces and operators and related concepts, but in addition to their work, we defined and study a number of additional concepts needed for the tensor product.

  Specifically: Hilbert-Schmidt and trace-class operators; compact operators; positive operators; the weak operator, strong operator, and weak* topology; the spectral theorem for compact operators; and the double commutant theorem.
\end{abstract}

\tableofcontents

% sane default for proof documents
\parindent 0pt\parskip 0.5ex

% generated text of all theories
\input{session}

% optional bibliography
\bibliographystyle{abbrv}
\bibliography{root}

\end{document}

%%% Local Variables:
%%% mode: latex
%%% TeX-master: t
%%% End:
