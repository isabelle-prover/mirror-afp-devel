\documentclass[fontsize=11pt,paper=a4,open=right,twoside,abstract=true]{scrreprt}
\usepackage{fixltx2e}
\usepackage{isabelle,isabellesym}
\usepackage[nocolortable, noaclist]{hol-ocl-isar}
\usepackage{booktabs}
\usepackage{graphicx}
\usepackage{amssymb}
\usepackage[numbers, sort&compress, sectionbib]{natbib}
\usepackage[caption=false]{subfig}
\usepackage{lstisar}
\usepackage{tabu}
\usepackage[]{mathtools}
\usepackage{prooftree}
\usepackage[english]{babel}
\usepackage[pdfpagelabels, pageanchor=false, plainpages=false]{hyperref}
% \usepackage[draft]{fixme}

% MathOCl expressions
\colorlet{MathOclColor}{Black}
\colorlet{HolOclColor}{Black}
\colorlet{OclColor}{Black}
%
\sloppy 

\uchyph=0
\graphicspath{{data/},{figures/}}
\allowdisplaybreaks

\renewcommand{\HolTrue}{\mathrm{true}}
\renewcommand{\HolFalse}{\mathrm{false}}
\newcommand{\ptmi}[1]{\using{\mi{#1}}}
\newcommand{\Lemma}[1]{{\color{BrickRed}%
    \mathbf{\operatorname{lemma}}}~\text{#1:}\quad}
\newcommand{\done}{{\color{OliveGreen}\operatorname{done}}}
\newcommand{\apply}[1]{{\holoclthykeywordstyle%
    \operatorname{apply}}(\text{#1})}
\newcommand{\fun} {{\holoclthykeywordstyle\operatorname{fun}}}
\newcommand{\definitionS} {{\holoclthykeywordstyle\operatorname{definition}}}
\newcommand{\where} {{\holoclthykeywordstyle\operatorname{where}}}
\newcommand{\datatype} {{\holoclthykeywordstyle\operatorname{datatype}}}
\newcommand{\types} {{\holoclthykeywordstyle\operatorname{types}}}
\newcommand{\pglabel}[1]{\text{#1}}
\renewcommand{\isasymOclUndefined}{\ensuremath{\mathtt{invalid}}}
\newcommand{\isasymOclNull}{\ensuremath{\mathtt{null}}}
\newcommand{\isasymOclInvalid}{\isasymOclUndefined}
\DeclareMathOperator{\inv}{inv}
\newcommand{\Null}[1]{{\ensuremath{\mathtt{null}_\text{{#1}}}}}
\newcommand{\testgen}{HOL-TestGen\xspace}
\newcommand{\HolOption}{\mathrm{option}}
\newcommand{\ran}{\mathrm{ran}}
\newcommand{\dom}{\mathrm{dom}}
\newcommand{\typedef}{\mathrm{typedef}}
\newcommand{\mi}[1]{\,\text{#1}}
\newcommand{\state}[1]{\ifthenelse{\equal{}{#1}}%
  {\operatorname{state}}%
  {\operatorname{\mathit{state}}(#1)}%
}
\newcommand{\mocl}[1]{\text{\inlineocl|#1|}}
\DeclareMathOperator{\TCnull}{null}
\DeclareMathOperator{\HolNull}{null}
\DeclareMathOperator{\HolBot}{bot}


% urls in roman style, theory text in math-similar italics
\urlstyle{rm}
\isabellestyle{it}
\newcommand{\ie}{i.\,e.\xspace}
\newcommand{\eg}{e.\,g.\xspace}
\renewcommand{\isamarkupheader}[1]{\chapter{#1}}
\renewcommand{\isamarkupsection}[1]{\section{#1}}
\renewcommand{\isamarkupsubsection}[1]{\subsection{#1}}
\renewcommand{\isamarkupsubsubsection}[1]{\subsubsection{#1}}
\renewcommand{\isamarkupsect}[1]{\section{#1}}
\renewcommand{\isamarkupsubsect}[1]{\susubsection{#1}}
\renewcommand{\isamarkupsubsubsect}[1]{\subsubsection{#1}}

\begin{document}
\renewcommand{\subsubsectionautorefname}{Section}
\renewcommand{\subsectionautorefname}{Section}
\renewcommand{\sectionautorefname}{Section}
\renewcommand{\chapterautorefname}{Chapter}
\newcommand{\subtableautorefname}{\tableautorefname}
\newcommand{\subfigureautorefname}{\figureautorefname}

\title{Featherweight OCL}
\subtitle{A Proposal for a Machine-Checked Formal Semantics for OCL 2.5}
\author{%
  \href{http://www.brucker.ch/}{Achim D. Brucker}\footnotemark[1]
  \and
  \href{https://www.lri.fr/~tuong/}{Fr\'ed\'eric Tuong}\footnotemark[3]
  \and
  \href{https://www.lri.fr/~wolff/}{Burkhart Wolff}\footnotemark[2]}
\publishers{%
  \footnotemark[1]~SAP AG, Vincenz-Priessnitz-Str. 1, 76131 Karlsruhe,
  Germany \texorpdfstring{\\}{} \href{mailto:"Achim D. Brucker"
    <achim.brucker@sap.com>}{achim.brucker@sap.com}\\[2em]
  %
  \footnotemark[3]~Univ. Paris-Sud, IRT SystemX, 8 av.~de la Vauve, \\
  91120 Palaiseau, France\\
  frederic.tuong@\{u-psud, irt-systemx\}.fr\\[2em]
  %
  \footnotemark[2]~Univ. Paris-Sud, Laboratoire LRI, UMR8623, 91405 Orsay, France\\
  CNRS, 91405 Orsay, France\texorpdfstring{\\}{}
  \href{mailto:"Burkhart Wolff" <burkhart.wolff@lri.fr>}{burkhart.wolff@lri.fr}
}


\maketitle

\begin{abstract}
  The Unified Modeling Language (UML) is one of the few modeling
  languages that is widely used in industry. While UML is mostly known
  as diagrammatic modeling language (\eg, visualizing class models),
  it is complemented by a textual language, called Object Constraint
  Language (OCL). OCL is a textual annotation language, based on a
  three-valued logic, that turns UML into a formal language.
  Unfortunately the semantics of this specification language, captured
  in the ``Annex A'' of the OCL standard, leads to different
  interpretations of corner cases.  Many of these corner cases had
  been subject to formal analysis since more than ten years.

  The situation complicated when with version 2.3 the OCL was aligned
  with the latest version of UML: this led to the extension of the
  three-valued logic by a second exception element, called
  \inlineocl{null}.  While the first exception element
  \inlineocl{invalid} has a strict semantics, \inlineocl{null} has a
  non strict semantic interpretation. These semantic difficulties lead
  to remarkable confusion for implementors of OCL compilers and
  interpreters.

  In this paper, we provide a formalization of the core of OCL in
  HOL\@. It provides denotational definitions, a logical calculus and
  operational rules that allow for the execution of OCL expressions by
  a mixture of term rewriting and code compilation.  Our formalization
  reveals several inconsistencies and contradictions in the current
  version of the OCL standard.  They reflect a challenge to define and
  implement OCL tools in a uniform manner.  Overall, this document is
  intended to provide the basis for a machine-checked text ``Annex A''
  of the OCL standard targeting at tool implementors.

\end{abstract}

\tableofcontents
\newpage
\section{Introduction}
 
\noindent
Building on Benjamin Bisping's research\cite{bens-algo}, 
we study (multi-weighted) energy games with reachability winning conditions. 
These are zero-sum two-player games with perfect information played on directed graphs 
labelled by (multi-weighted) energy functions. 

Bisping~\cite{bens-algo} introduces a class of energy games, called \textit{declining energy games} 
and provides an algorithm to compute minimal attacker winning budgets (i.e. Pareto fronts).
He claims decidability of this class of energy games if the set of positions is finite.
We substantiate this claim by providing a formal proof using a simplyfied and generalised version 
of that algorithm~\cite{Lemke2024}.

We abstract the necessary properties used in the proof and introduce a new 
class of energy games: Galois energy games. In such games updates can be 
undone through Galois connections, yielding a weakened form of inversion 
sufficient for an algorihm similar to standard shortest path algorithms.
We estabish decidability of the unknown and known initial credit problem for Galois energy games 
over well-founded bounded join-semilattices with a finite set of positions.

Galois energy games can be instantiated to common energy games, declining energy games~\cite{bens-algo}, multi-weighted reachability games~\cite{lexicograph} and  
coverability on vector addition systems with states~\cite{kunnemann2023coverability}. 
By confirming a subclass relationship (via sublocales) we conclude decidability of Galois energy games 
over vectors of (extended) naturals with the component-wise order. Finally, we show this in the case 
of vector-addition and min-updates only, subsuming the case of Bisping's declining energy games.

For a broader perspective on the formalised results, including motivation, a high-level proof outline, 
complexity considerations, and connections to related work, we refer to the preprint~\cite{preprint}.

\subsection*{Theory Structure}

We now give an overview of all our theories.
In summary, we first formalise energy games with reachability winning conditions (in Energy\_Game.thy), 
then formalise Galois energy games (in Galois\_Energy\_Game.thy) and prove decidability (in Decidability.thy).
Finally, we formalise a superclass of Bisping's declining energy games (in Natural\_Galois\_Energy\_Game.thy) and conclude decidability.

The file strucrture is given by the following excerpt of the session graph, where the theories above are imported by the ones below.

\begin{figure}[H]
\begin{center}

\definecolor{gray245}{RGB}{245, 245, 245}
\definecolor{color0}{RGB}{0, 0, 0}
\definecolor{color1}{RGB}{51, 51, 51}

\tikzstyle{rect} = [rectangle, minimum width=2.4cm, minimum height=1cm, text centered, font=\normalsize, color=color1, draw=color0, line width=1, fill=gray245]
\tikzstyle{arrowdefi} = [thick, draw=color1, line width=2, ->, >=stealth]

\begin{tikzpicture}[node distance=2cm]
\node (bisping) [state, rect, text width=5cm] {Natural\_Galois\_Energy\_Game};
\node (updates) [state, rect, above of=bisping, xshift=+2.8cm, text width=4cm] {Update};
\node (order) [state, rect, above of=updates, text width=4cm] {Energy\_Order};
\node (decidable) [state, rect, above of=bisping, xshift=-2.8cm, text width=4cm] {Decidability};
\node (galois) [state, rect, above of=decidable, text width=4cm] {Galois\_Energy\_Game};
\node (games) [state, rect, above of=galois, text width=4cm] {Energy\_Game};
\node (list) [state, rect, above of=order, text width=4cm] {List\_Lemmas};

\draw 
(order) -- (updates)
(updates) -- (bisping)
(games) -- (galois)
(galois) -- (decidable)
(bisping) -- (decidable)
(list) -- (order)
;
\end{tikzpicture}
\end{center}
\end{figure}

Energy games are formalised as two-player zero-sum games with perfect information and reachability winning conditions played on labeled directed graphs in Energy\_Game.thy. 
In particular, strategies and an inductive characterisation of winning budgets is discussed.
(This corresponds to section 2.1 and 2.2 in the preprint~\cite{preprint}.)

Galois energy games over well-founded bounded join-semilattices are formalized in Galois\_Energy\_Game.thy. 
(This corresponds to section 2.3 in the preprint~\cite{preprint}.)

In Decidability.thy we formalise one iteration of a simplyfied and generalised version of Bisping's algorithm. 
Using an order on possible Pareto fronts we are able to apply Kleene's fixed point theorem. 
Assuming the game graph to be finite we then prove correctness of the algorithm. Further, we provide the key argument for termination, thus proving decidability of Galois energy games.
(This corresponds to section 3.2 in the preprint~\cite{preprint}.)

The file List\_Lemmas.thy contains a few simple observations about lists, specifically when using \texttt{those}. This file's contents can be found in the appendix.

In Energy\_Order.thy we introduce the energies, i.e.\ vectors with entries in the extended natural numbers, and the component-wise order. There we establish that this order is a well-founded bounded join-semilattice. 

In Update.thy we define a superset of Bisping's updates. These are partial functions of energy vectors updating each component by subtracting or adding one, replacing it with the minimum of some components or not changing it. In particular, we observe that these functions are monotonic and have upward-closed domains.
Further, we introduce a generalisation of Bisping's inversion and relate it to the updates using Galois connections. 

In Natural\_Galois\_Energy\_Game.thy we formalise galois energy games over the previously defined with a fixed dimension. 
Afterwards, we formalise a subclass of such games where all edges of the game graph are labeled with a representation of the previously discussed updates (and thereby formalise Bisping's declining energy games).
Finally, we establish the subclass-relationships and thereby conclude decidability. 
(This corresponds to section 4.2 in the preprint~\cite{preprint}.)

\part{A Proposal for Formal Semantics of OCL 2.5}


\input{OCL_core.tex}
\input{OCL_lib.tex}
\input{OCL_state.tex}
\input{OCL_tools.tex}
\input{OCL_main.tex}


\part{Examples}
\chapter{The Employee Analysis Model}
\label{ex:employee-analysis}
\input{Employee_AnalysisModel_UMLPart.tex}
\input{Employee_AnalysisModel_OCLPart.tex}
\chapter{The Employee Design Model}
\label{ex:employee-design}
\input{Employee_DesignModel_UMLPart.tex}
\input{Employee_DesignModel_OCLPart.tex}


%%% Local Variables: 
%%% mode: latex
%%% TeX-master: "root"
%%% End:

\include{conclusion}
\bibliographystyle{abbrvnat}
\bibliography{root} 

\end{document}

%%% Local Variables:
%%% mode: latex
%%% TeX-master: t
%%% End:

%  LocalWords:  implementors denotational OCL UML
