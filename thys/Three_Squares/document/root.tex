\documentclass[11pt, a4paper]{article}
\usepackage[T1]{fontenc}
\usepackage{isabelle, isabellesym}
\usepackage{amssymb} % for \<nexists>
\usepackage{pdfsetup}

% urls in roman style, theory text in math-similar italics
\urlstyle{rm}
\isabellestyle{it}

% for uniform font size
% \renewcommand{\isastyle}{\isastyleminor}

\begin{document}

\title{Three squares theorem}
\author{Anna Danilkin, Loïc Chevalier}
\maketitle

\begin{abstract}
  We formalize the Legendre's three squares theorem and its consequences,
  in particular the following results:
  \begin{enumerate}
    \item A natural number can be represented as the sum of
          three squares of natural numbers if and only if it is not
          of the form $4^a (8 k + 7)$, where $a$ and $k$ are natural numbers.
    \item If $n$ is a natural number such that $n \equiv 3 \pmod{8}$,
          then $n$ can be be represented as the sum of three squares
          of odd natural numbers.
  \end{enumerate}

  Consequences include the following:
  \begin{enumerate}
    \item An integer $n$ can be written as $n = x^2 + y^2 + z^2 + z$,
          where $x$, $y$, $z$ are integers, if and only if $n \geq 0$.
    \item The Legendre's four squares theorem: any natural number
          can be represented as the sum of four squares of natural numbers.
  \end{enumerate}
\end{abstract}

We follow the book of Melvyn B. Nathanson
`Additive Number Theory: The Classical Bases' \cite{nathanson1996}.

We plan to make use of the first consequence mentioned above in an
upcoming AFP entry on Diophantine equations. More concretely, we intend
to formalize universal pairs over the integers which requires expressing
a natural number as a polynomial in integers while only using few variables.

\tableofcontents

% sane default for proof documents
\parindent 0pt\parskip 0.5ex

% generated text of all theories
\input{session}

% bibliography
\bibliographystyle{abbrv}
\bibliography{root}

\end{document}
