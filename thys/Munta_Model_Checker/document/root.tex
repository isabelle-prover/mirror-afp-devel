\documentclass[10pt,a4paper]{article}
\usepackage[T1]{fontenc}
\usepackage{isabelle,isabellesym}
\usepackage{a4wide}
\usepackage[english]{babel}

% further packages required for unusual symbols (see also
% isabellesym.sty), use only when needed

\usepackage{amssymb}
  %for \<leadsto>, \<box>, \<diamond>, \<sqsupset>, \<mho>, \<Join>,
  %\<lhd>, \<lesssim>, \<greatersim>, \<lessapprox>, \<greaterapprox>,
  %\<triangleq>, \<yen>, \<lozenge>
\usepackage{wasysym}
% for \<hole>

% this should be the last package used
\usepackage{pdfsetup}

% urls in roman style, theory text in math-similar italics
\urlstyle{rm}
\isabellestyle{literal}

% for uniform font size
\renewcommand{\isastyle}{\isastyleminor}

\renewcommand{\isamarkupsection}[1]{\section{#1}}
\renewcommand{\isamarkupsubsection}[1]{\subsection{#1}}
\renewcommand{\isamarkupsubsubsection}[1]{\paragraph{#1}}

\newcommand{\munta}{\textsc{Munta}}

\begin{document}

\title{\munta: A Verified Model Checker for Timed Automata}
\author{Simon Wimmer}
\date{}

\maketitle
\begin{abstract}
  \munta\ is a verified model checker for timed automata.
  It has been described in detail in a PhD thesis \cite{wimmer-phd}
  and conference publications~\cite{munta-tacas,munta-tool-paper}.
  This entry supersedes earlier versions of \munta\ hosted on GitHub.
\end{abstract}

\setcounter{tocdepth}{2}
\tableofcontents
\newpage

\section*{Introduction}

\munta\ is a verified model checker for timed automata.
It has been described in detail in a PhD thesis \cite{wimmer-phd}
and presented in peer-reviewed conference publications~\cite{munta-tacas,munta-tool-paper}.

This AFP entry builds upon and extends formalizations from the following previous entries:
\begin{itemize}
  \item \texttt{Timed\_Automata}~\cite{timed-automata-afp}
  \item \texttt{Worklist\_Algorithms}~\cite{worklist-afp}
  \item \texttt{Difference\_Bound\_Matrices}~\cite{dbm-afp}
\end{itemize}
and supersedes an earlier version of \munta\ hosted on GitHub:
\url{https://github.com/wimmers/munta}.

\paragraph{Usage}
Theory \texttt{Simple\_Network\_Language\_Export\_Code} (page \pageref{export-munta}) describes
how to obtain executable code for and run \munta\ from within Isabelle.

This entry also provides a build setup for the \textsc{munta} executable
using the MLton SML compiler.
Build and run instructions are given in theory \texttt{Munta\_Compile\_MLton}
(on page \pageref{build-mlton}).
A number of example model files can be found in the \texttt{benchmarks} directory.

\paragraph{Graphical Interface} 
A graphical user interface for \textsc{munta} is also available:
\url{https://munta.isabelle.systems}.
The GUI frontend can communicate with a local instance of \munta\ behind a Python-based server
or run \munta\ directly in the browser.
Detailed run instructions are given on Github:
\url{https://github.com/wimmers/munta/?tab=readme-ov-file#graphical-user-interface}.

\paragraph{OCaml Code Generation} 
While the default backend targets SML/MLton,
\textsc{munta} in principle also supports code generation to OCaml
(c.f.\ theory \texttt{Simple\_Network\_Language\_Export\_Code} on page \pageref{export-ocaml}).
Among its use cases is the browser-based version of the GUI, which is compiled from the OCaml code.
For instructions on compiling with the \texttt{dune} build system, see the \texttt{ocaml} branch:
\url{https://github.com/wimmers/munta/tree/ocaml}.

\newpage

% sane default for proof documents
\parindent 0pt\parskip 0.5ex

% generated text of all theories
\input{session}

% optional bibliography
\bibliographystyle{abbrv}
\bibliography{root}

\end{document}

%%% Local Variables:
%%% mode: latex
%%% TeX-master: t
%%% End:

