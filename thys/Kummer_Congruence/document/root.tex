\documentclass[11pt,a4paper]{article}
\usepackage[T1]{fontenc}
\usepackage{isabelle,isabellesym}
\usepackage{amsfonts, amsmath, amssymb}

% this should be the last package used
\usepackage{pdfsetup}
\usepackage[shortcuts]{extdash}

% urls in roman style, theory text in math-similar italics
\urlstyle{rm}
\isabellestyle{rm}


\begin{document}

\title{Congruences of Bernoulli Numbers}
\author{Manuel Eberl}
\maketitle

\begin{abstract}
This entry provides proofs for two important congruences involving Bernoulli numbers.
The proofs follow Cohen's textbook \emph{Number Theory Volume II: Analytic and Modern Tools}~\cite{cohen2007}.
In the following we write $\mathcal{B}_k = N_k / D_k$ for the $k$-th Bernoulli number (with $\text{gcd}(N_k, D_k) = 1$).

The first result that I showed is \emph{Voronoi's congruence}, which states that for any even integer
$k\geq 2$ and all positive coprime integers $a$, $n$ we have:
\[(a^k - 1) N_k \equiv k a^{k-1} D_k \sum_{m=1}^{n-1} m^{k-1} \left\lfloor\frac{ma}{n}\right\rfloor
\hspace*{3mm}(\text{mod}\ n)\]

Building upon this, I then derive \emph{Kummer's congruence}. In its common form, it states that 
for a prime $p$ and even integers $k,k'$ with $\text{min}(k,k')\geq e+1$ and $(p - 1) \nmid k$ and
$k \equiv k'\ (\text{mod}\ \varphi(p^e))$, we have:
\[\hspace*{6mm}\frac{\mathcal{B}_k}{k} \equiv \frac{\mathcal{B}_{k'}}{k'}\hspace*{3mm}(\text{mod}\ p^e)\]
The version proved in my entry is slightly more general than this.

One application of these congruences is to prove that there are infinitely many irregular primes,
which I formalised as well.
\end{abstract}

\newpage
\tableofcontents

\newpage
\parindent 0pt\parskip 0.5ex

\input{session}

\raggedright
\bibliographystyle{abbrv}
\bibliography{root}

\end{document}

