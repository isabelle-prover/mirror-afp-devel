\documentclass[11pt,a4paper]{article}
\usepackage[T1]{fontenc}
\usepackage{isabelle,isabellesym}

\usepackage{amssymb}
  %for \<leadsto>, \<box>, \<diamond>, \<sqsupset>, \<mho>, \<Join>,
  %\<lhd>, \<lesssim>, \<greatersim>, \<lessapprox>, \<greaterapprox>,
  %\<triangleq>, \<yen>, \<lozenge>

\usepackage{latexsym}
\usepackage{wasysym}

\usepackage[only,bigsqcap,bigparallel,fatsemi,interleave,sslash]{stmaryrd}
  %for \<Sqinter>, \<Parallel>, \<Zsemi>, \<Parallel>, \<sslash>

% this should be the last package used
\usepackage{pdfsetup}

% urls in roman style, theory text in math-similar italics
\urlstyle{rm}
\isabellestyle{it}

\begin{document}

\title{Faithful Logic Embeddings in HOL --- Deep and Shallow (Isabelle/HOL dataset)}
\author{Christoph Benzm{\"u}ller}
\maketitle

\begin{abstract}
A recipe for the simultaneous deployment of different forms of deep and shallow embeddings of non-classical logics in classical higher-order logic is presented, which enables interactive or even automated faithfulness proofs between the logic embeddings. The approach, which is particularly fruitful for logic education, is explained in detail in an associated CADE conference paper. This paper presents the corresponding Isabelle/HOL dataset (which is only slightly modified to meet AFP requirements).
\end{abstract}

\tableofcontents

% sane default for proof documents
\parindent 0pt\parskip 0.5ex


\section{Introduction}
The Isabelle/HOL dataset associated with \cite{CADEversion} is presented.  Sections \ref{sec:pmlinhol_deep}, \ref{sec:pmlinhol_shallow_max} and \ref{sec:pmlinhol_shallow_min} present deep, maximally shallow, and minimally shallow embeddings of propositional modal logic (PML) in classical higher-order logic (HOL). These are connected,  as a novel contribution, by automated faithfulness proofs given in Sect.~\ref{sec:faithfulness}. This connection ensures that these deep and shallow embeddings can now be used interchangeably in subsequent applications.
Several experiments with the presented embeddings are presented in Sect.~\ref{sec:automation_tests}. The presented work is conceptual in nature and can be adapted to other non-classical logics. For more detailed explanations of the presented material, including a discussion of related works, see \cite{CADEversion}.

% generated text of all theories
\input{session}

% optional bibliography
\bibliographystyle{abbrv}
\bibliography{root}

\end{document}

%%% Local Variables:
%%% mode: latex
%%% TeX-master: t
%%% End:
