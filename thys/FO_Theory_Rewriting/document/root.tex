\documentclass[11pt,a4paper]{article}
\usepackage[T1]{fontenc}
\usepackage{isabelle,isabellesym}

\usepackage{url}
\usepackage{amssymb}
\usepackage{xspace}

% this should be the last package used
\usepackage{pdfsetup}

% urls in roman style, theory text in math-similar italics
\urlstyle{rm}
\isabellestyle{it}

\newcommand\isafor{\textsf{Isa\kern-0.15exF\kern-0.15exo\kern-0.15exR}}
\newcommand\ceta{\textsf{C\kern-0.15exe\kern-0.45exT\kern-0.45exA}}

\begin{document}


\title{A Formalization of the First Order Theory of Rewriting (FORT) \footnote{Supported by FWF (Austrian Science Fund) project P30301.}}
\author{Alexander Lochmann \and Bertram Felgenhauer}
\maketitle

\begin{abstract}
The first-order theory of rewriting (FORT) is a decidable theory for
linear variable-separated rewrite systems. The decision
procedure is based on tree automata technique and an inference system presented in \cite{MLMF21}.
This AFP entry provides a formalization of the underlying decision procedure.
Moreover it allows to generate a function that can verify each inference step via the
code generation facility of Isabelle/HOL.

Additionally it contains the specification of a certificate language
(that allows to state proofs in FORT) and a formalized function that
allows to verify the validity of the proof.
This gives software tool authors, that implement the decision procedure,
the possibility to verify their output.
\end{abstract}

\tableofcontents

\section{Introduction}

The first-order theory of rewriting (FORT) is a fragment of
first-order predicate logic with predefined predicates.
The language allows to state many interesting properties of
term rewrite systems and is decidable for left-linear right-ground systems.
This was proven by Dauchet and Tison \cite{DT90}.

In this AFP entry we provide a formalized proof of an improved decision procedure
for the first-order theory of rewriting. We introduce basic definitions to
represent the rewrite semantics and connect FORT to first-order logic
via the AFP entry "First-Order Logic According to Fitting" by Stefan Berghofer \cite{B07}.
To prove the decidability and more importantly to allow code generation a relation between
formulas in FORT and regular tree language is constructed. The tree language
contains all witnesses of free variables satisfying the formula, details can be found in \cite{LMMF21}.

Moreover we present a certificate language which is rich enough to express the various au-
tomata operations in decision procedures for the first-order theory of rewrit-
ing as well as numerous predicate symbols that may appear in formulas in
this theory, for more details see \cite{MLMF21}.

\input{session}

\bibliographystyle{abbrv}
\bibliography{root}

\end{document}

