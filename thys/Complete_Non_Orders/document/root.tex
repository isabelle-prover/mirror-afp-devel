\documentclass[11pt,a4paper]{article}
\usepackage[T1]{fontenc}
\usepackage{isabelle,isabellesym}
\usepackage{amssymb,amsmath,stmaryrd}
\usepackage{tikz}
\usetikzlibrary{backgrounds}
\usetikzlibrary{positioning}
\usetikzlibrary{shapes}

% this should be the last package used
\usepackage{pdfsetup}

% urls in roman style, theory text in math-similar italics
\urlstyle{rm}
\isabellestyle{it}

\newcommand\SLE{\sqsubseteq}
\newcommand\Nat{\mathbb{N}}
\newcommand\SLT{\sqsubset}
\newcommand\SUP{\bigsqcup}


\makeatletter

\def\tp@#1#2{\@ifnextchar[{\tp@@{#1}{#2}}{\tp@@@{#1}{#2}}}
\def\tp@@#1#2[#3]#4{#3#1\def\mid{\mathrel{#3|}}#4#3#2}
\def\tp@@@#1#2#3{\bgroup\left#1\def\mid{\;\middle|\;}#3\right#2\egroup}
\def\pa{\tp@()}
\def\tp{\tp@\langle\rangle}
\def\set{\tp@\{\}}

\makeatother


\begin{document}

\title{Complete Non-Orders and Fixed Points}
\author{Akihisa Yamada and J\'er\'emy Dubut}
\maketitle

\begin{abstract}
We develop an Isabelle/HOL library
of order-theoretic concepts, such as various completeness conditions and
fixed-point theorems.
We keep our formalization as general as possible:
we reprove several well-known results about complete orders,
often with only antisymmetry or attractivity, a mild condition implied by either 
antisymmetry or transitivity.
In particular, we generalize various theorems ensuring
the existence of a quasi-fixed point of monotone maps over complete relations,
and show that the set of (quasi-)fixed points is itself complete.
This result generalizes and strengthens theorems of Knaster--Tarski, Bourbaki--Witt, Kleene, Markowsky, Pataraia, Mashburn, Bhatta--George, and Stouti--Maaden.
\end{abstract}

\tableofcontents

\section{Introduction}

The main driving force towards mechanizing mathematics using proof assistants
% For a journal version:
%such as Coq~\cite{coq}, Agda~\cite{agda}, HOL-Light~\cite{hol-light}, and Isabelle/HOL~\cite{isabelle},
has been the reliability they offer,
%Once a theorem is formally proved using a proof assistant,
%we can certainly trust the theorem without understanding how it is proved,
%provided the claim itself is correctly formalized.
%The most prominent of such examples are
%the four-color theorem proved using Coq~\cite{4color}, and
%the Kepler theorem proved with HOL-Light and Isabelle/HOL~\cite{flyspeck}.
%Moreover the reliability of proof assistants is utilized for
%developing bug-free software, for instance,
%the CompCert C compliler~\cite{compcert},
%the seL4 OS kernel~\cite{sel4},
%the IsaFoR/CeTA certifier for validating program analyzers~\cite{isafor},
%...
%
exemplified prominently by~\cite{4color},~\cite{flyspeck},~\cite{sel4}, etc.
In this work, we utilize another aspect of proof assistants:
they are also engineering tools for developing mathematical theories.

\emph{Fixed-point theorems} are important in computer science, such as 
in denotational semantics \cite{scott71} and
in abstract interpretation \cite{cousot77}, as they allow the definition 
of semantics of loops and recursive functions.
The Knaster--Tarski theorem~\cite{tarski55}
shows that
any monotone map $f : A \to A$ over complete lattice $(A,\SLE)$ has a fixed point,
and the set of fixed points forms also a complete lattice.
The result was generalized in various ways:
Markowsky~\cite{markowsky76}
showed a corresponding result for \emph{chain-complete} posets.
The proof uses
the Bourbaki--Witt theorem~\cite{bourbaki49},
stating that any inflationary map over a chain-complete poset has a fixed point.
The original proof of the latter is non-elementary in
the sense that it relies on ordinals and Hartogs' theorem.
Pataraia~\cite{pataraia97}
gave an elementary proof that
monotone maps over \emph{pointed directed-complete} poset
has a fixed point.
Fixed points are studied also for \emph{pseudo-orders}~\cite{trellis},
relaxing transitivity.
Stouti and Maaden~\cite{SM13} showed that every monotone map over a complete pseudo-order has a (least) fixed point.
Markowsky's result was also generalized to
\emph{weak} chain-complete pseudo-orders by Bhatta and George~\cite{Bhatta05,BG11}.

Another line of order-theoretic fixed points is the \emph{iterative} approach.
Kantorovitch showed that
for \emph{$\omega$-continuous} map $f$
over a complete lattice,\footnote{
More precisely,
he assumes a conditionally complete lattice defined over vectors and that
$\bot \SLE f\:\bot$ and $f\:v' \SLE v'$.
Hence $f$, which is monotone,
is a map over the complete lattice $\{v \mid \bot \SLE v \SLE v'\}$.
}
the
iteration $\bot, f\:\bot,f^2\:\bot,\dots$ converges to a fixed point
\cite[Theorem I]{kantorovitch39}.
Tarski~\cite{tarski55} also claimed a similar result for a
\emph{countably distributive} map over a
\emph{countably complete Boolean algebra}.
Kleene's fixed-point theorem states that,
for \emph{Scott-continuous} maps over pointed directed-complete posets,
the iteration converges to the least fixed point.
Finally, Mashburn~\cite{mashburn83} proved a version for 
$\omega$-continuous maps over $\omega$-complete posets,
which covers Kantorovitch's, Tarski's and Kleene's claims.



%In this paper,
%we formalize these fixed-point theorems in Isabelle/HOL.
%Here we adopt an \emph{as-general-as-possible} approach:
%all theorems
%are proved without assuming the underlying relations to be orders.
%One can easily find several formalizations of complete partial orders or lattices in Isabelle's standard library.
%They are, however, defined on partial orders
%and thus not directly reusable for general relations.

In particular, we provide the following:
\begin{itemize}
\item Several \emph{locales} that help organizing the different order-theoretic conditions, 
such as reflexivity, transitivity, antisymmetry, and their combination, as well as concepts such as connex and well-related sets, analogues of chains 
and well-ordered sets in a non-ordered context.
%(\prettyref{sec:prelim}).
\item Existence of fixed points:
We provide two proof schemes to prove that monotone or inflationary mapping 
$f : A \to A$ over a complete related set $\tp{A,\SLE}$
has a \emph{quasi-fixed point} $f\:x \sim x$,
meaning $x \SLE f\:x \mathrel\land f\:x \SLE x$, for various notions of completeness.
The first one,
%(\prettyref{sec:knaster-tarski})
similar to the original proof by Tarski \cite{tarski55}, 
does not require any ordering assumptions,
but relies on completeness with respect to all subsets.
The second one,
%(\prettyref{sec:weak_chain}),
inspired by a \emph{constructive} approach by Grall \cite{grall10},
is a proof scheme based on the notion of derivations.
Here we demand antisymmetry (to avoid the necessity of the axiom of choice),
but can be instantiated to \emph{well-complete} sets,
a generalization of weak chain-completeness.
This also allows us to generalize Bourbaki--Witt theorem \cite{bourbaki49} to pseudo-orders.
\item Completeness of the set of fixed points:
%(\prettyref{sec:completeness}): We further show that
if $(A,\SLE)$ satisfies a mild condition, which we call \emph{attractivity} and
which is implied by either transitivity or antisymmetry,
then the set of quasi-fixed points inherits the completeness class from
$(A,\SLE)$, if it is at least well-complete.
The result instantiates to
the full completeness (generalizing Knaster--Tarski and \cite{SM13}), 
directed-completeness \cite{pataraia97},
chain-completeness \cite{markowsky76}, 
and weak chain-completeness \cite{BG11}.
\item Iterative construction:
% (\prettyref{sec:kleene}):
For an $\omega$-continuous map over an $\omega$-complete related set,
we show that suprema of $\set{f^n\:\bot \mid n\in\Nat}$ are quasi-fixed points.
Under attractivity, the quasi-fixed points obtained from this scheme 
are precisely the least quasi-fixed points of $f$.
This generalizes Mashburn's result, and thus ones by
Kantorovitch, Tarski and Kleene.
\end{itemize}
We remark that all these results would have required much more effort than we spent
(if possible at all),
if we were not with the aforementioned smart assistance by Isabelle.
Our workflow was often the following: first we formalize existing proofs, try relaxing assumptions, 
see where proof breaks, and at some point ask for a counterexample.

Concerning Isabelle formalization, one can easily find several formalizations 
of complete partial orders or lattices in Isabelle?s standard library. They are, 
however, defined on partial orders, either in form of classes or locales, and 
thus not directly reusable for non-orders. Nevertheless we tried to make our 
formalization compatible with the existing ones, and various
correspondences are ensured.

This archive is the third version of this work.
The first version has been published in
the conference paper \cite{yamada19}.
The second version has been published in the 
journal paper \cite{DubutY21}.
The third version is a restructuration of the second version
for future formalizations, including \cite{yamada23}.

% include generated text of all theories
\input{session}

\bibliographystyle{abbrv}
\bibliography{root}

\end{document}
