\newpage
\section{Introduction}
 
\noindent
Building on Benjamin Bisping's research\cite{bens-algo}, 
we study (multi-weighted) energy games with reachability winning conditions. 
These are zero-sum two-player games with perfect information played on directed graphs 
labelled by (multi-weighted) energy functions. 

Bisping~\cite{bens-algo} introduces a class of energy games, called \textit{declining energy games} 
and provides an algorithm to compute minimal attacker winning budgets (i.e. Pareto fronts).
He claims decidability of this class of energy games if the set of positions is finite.
We substantiate this claim by providing a formal proof using a simplyfied and generalised version 
of that algorithm~\cite{Lemke2024}.

We abstract the necessary properties used in the proof and introduce a new 
class of energy games: Galois energy games. In such games updates can be 
undone through Galois connections, yielding a weakened form of inversion 
sufficient for an algorihm similar to standard shortest path algorithms.
We estabish decidability of the unknown and known initial credit problem for Galois energy games 
over well-founded bounded join-semilattices with a finite set of positions.

Galois energy games can be instantiated to common energy games, declining energy games~\cite{bens-algo}, multi-weighted reachability games~\cite{lexicograph} and  
coverability on vector addition systems with states~\cite{kunnemann2023coverability}. 
By confirming a subclass relationship (via sublocales) we conclude decidability of Galois energy games 
over vectors of (extended) naturals with the component-wise order. Finally, we show this in the case 
of vector-addition and min-updates only, subsuming the case of Bisping's declining energy games.

For a broader perspective on the formalised results, including motivation, a high-level proof outline, 
complexity considerations, and connections to related work, we refer to the preprint~\cite{preprint}.

\subsection*{Theory Structure}

We now give an overview of all our theories.
In summary, we first formalise energy games with reachability winning conditions (in Energy\_Game.thy), 
then formalise Galois energy games (in Galois\_Energy\_Game.thy) and prove decidability (in Decidability.thy).
Finally, we formalise a superclass of Bisping's declining energy games (in Natural\_Galois\_Energy\_Game.thy) and conclude decidability.

The file strucrture is given by the following excerpt of the session graph, where the theories above are imported by the ones below.

\begin{figure}[H]
\begin{center}

\definecolor{gray245}{RGB}{245, 245, 245}
\definecolor{color0}{RGB}{0, 0, 0}
\definecolor{color1}{RGB}{51, 51, 51}

\tikzstyle{rect} = [rectangle, minimum width=2.4cm, minimum height=1cm, text centered, font=\normalsize, color=color1, draw=color0, line width=1, fill=gray245]
\tikzstyle{arrowdefi} = [thick, draw=color1, line width=2, ->, >=stealth]

\begin{tikzpicture}[node distance=2cm]
\node (bisping) [state, rect, text width=5cm] {Natural\_Galois\_Energy\_Game};
\node (updates) [state, rect, above of=bisping, xshift=+2.8cm, text width=4cm] {Update};
\node (order) [state, rect, above of=updates, text width=4cm] {Energy\_Order};
\node (decidable) [state, rect, above of=bisping, xshift=-2.8cm, text width=4cm] {Decidability};
\node (galois) [state, rect, above of=decidable, text width=4cm] {Galois\_Energy\_Game};
\node (games) [state, rect, above of=galois, text width=4cm] {Energy\_Game};
\node (list) [state, rect, above of=order, text width=4cm] {List\_Lemmas};

\draw 
(order) -- (updates)
(updates) -- (bisping)
(games) -- (galois)
(galois) -- (decidable)
(bisping) -- (decidable)
(list) -- (order)
;
\end{tikzpicture}
\end{center}
\end{figure}

Energy games are formalised as two-player zero-sum games with perfect information and reachability winning conditions played on labeled directed graphs in Energy\_Game.thy. 
In particular, strategies and an inductive characterisation of winning budgets is discussed.
(This corresponds to section 2.1 and 2.2 in the preprint~\cite{preprint}.)

Galois energy games over well-founded bounded join-semilattices are formalized in Galois\_Energy\_Game.thy. 
(This corresponds to section 2.3 in the preprint~\cite{preprint}.)

In Decidability.thy we formalise one iteration of a simplyfied and generalised version of Bisping's algorithm. 
Using an order on possible Pareto fronts we are able to apply Kleene's fixed point theorem. 
Assuming the game graph to be finite we then prove correctness of the algorithm. Further, we provide the key argument for termination, thus proving decidability of Galois energy games.
(This corresponds to section 3.2 in the preprint~\cite{preprint}.)

The file List\_Lemmas.thy contains a few simple observations about lists, specifically when using \texttt{those}. This file's contents can be found in the appendix.

In Energy\_Order.thy we introduce the energies, i.e.\ vectors with entries in the extended natural numbers, and the component-wise order. There we establish that this order is a well-founded bounded join-semilattice. 

In Update.thy we define a superset of Bisping's updates. These are partial functions of energy vectors updating each component by subtracting or adding one, replacing it with the minimum of some components or not changing it. In particular, we observe that these functions are monotonic and have upward-closed domains.
Further, we introduce a generalisation of Bisping's inversion and relate it to the updates using Galois connections. 

In Natural\_Galois\_Energy\_Game.thy we formalise galois energy games over the previously defined with a fixed dimension. 
Afterwards, we formalise a subclass of such games where all edges of the game graph are labeled with a representation of the previously discussed updates (and thereby formalise Bisping's declining energy games).
Finally, we establish the subclass-relationships and thereby conclude decidability. 
(This corresponds to section 4.2 in the preprint~\cite{preprint}.)
