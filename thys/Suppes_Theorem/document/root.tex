\documentclass[11pt,a4paper]{report}
\usepackage[T1]{fontenc}
\usepackage{isabelle, isabellesym}
\usepackage{amsfonts, amsmath, amssymb}
\usepackage{mathpartir}
\usepackage{eurosym}

\usepackage[only,bigsqcap]{stmaryrd}

% this should be the last package used
\usepackage{pdfsetup}

% urls in roman style, theory text in math-similar italics
\urlstyle{rm}
\isabellestyle{it}

\begin{document}

\title{Suppes' Theorem For Probability Logic}
\author{Matthew Doty}

\maketitle

\begin{abstract}
  We develop finitely additive probability logic and prove a theorem
  of Patrick Suppes that asserts that $\Psi \vdash \phi$ in classical
  propositional logic if and only if
  $(\sum \psi \leftarrow \Psi.\; 1 - \mathcal{P} \psi) \geq 1 -
  \mathcal{P} \phi$ holds for all probabilities $\mathcal{P}$. We also
  provide a novel \emph{dual} form of Suppes' Theorem, which holds
  that
  $(\sum \phi \leftarrow \Phi.\; \mathcal{P} \phi) \leq \mathcal{P}
  \psi$ for all probabilities $\mathcal{P}$ if and only
  $\left(\bigvee \Phi\right) \vdash \psi$ and all of the formulae in
  $\Phi$ are logically exclusive from one another. Our proofs use
  \emph{Maximally Consistent Sets}, and as a consequence, we obtain
  two \emph{collapse} theorems. In particular, we show
  $(\sum \phi \leftarrow \Phi.\; \mathcal{P} \phi) \geq \mathcal{P}
  \psi$ holds for all probabilities $\mathcal{P}$ if and only if
  $(\sum \phi \leftarrow \Phi.\; \delta\; \phi) \geq \delta\; \psi$
  holds for all binary-valued probabilities $\delta$, along with the
  dual assertion that
  $(\sum \phi \leftarrow \Phi. \;\mathcal{P} \phi) \leq \mathcal{P}
  \psi$ holds for all probabilities $\mathcal{P}$ if and only if
  $(\sum \phi \leftarrow \Phi.\; \delta\; \phi) \leq \delta\; \psi$
  holds for all binary-valued probabilities $\delta$.
\end{abstract}

\tableofcontents

\newpage

\parindent 0pt\parskip 0.5ex

\input{session}

\bibliographystyle{abbrv}
\bibliography{root}

\end{document}

%%% Local Variables:
%%% mode: latex
%%% TeX-master: t
%%% End:
