\documentclass[11pt,a4paper]{article}
\usepackage{isabelle,isabellesym}

% further packages required for unusual symbols (see also
% isabellesym.sty), use only when needed

\usepackage{pifont}

\usepackage{mathpartir}

\usepackage{amsmath}

\usepackage{amssymb}
  %for \<leadsto>, \<box>, \<diamond>, \<sqsupset>, \<mho>, \<Join>,
  %\<lhd>, \<lesssim>, \<greatersim>, \<lessapprox>, \<greaterapprox>,
  %\<triangleq>, \<yen>, \<lozenge>

%\usepackage{eurosym}
  %for \<euro>

%\usepackage[only,bigsqcap,bigparallel,fatsemi,interleave,sslash]{stmaryrd}
  %for \<Sqinter>, \<Parallel>, \<Zsemi>, \<Parallel>, \<sslash>

%\usepackage{eufrak}
  %for \<AA> ... \<ZZ>, \<aa> ... \<zz> (also included in amssymb)

%\usepackage{textcomp}
  %for \<onequarter>, \<onehalf>, \<threequarters>, \<degree>, \<cent>,
  %\<currency>

% this should be the last package used
\usepackage{pdfsetup}

% urls in roman style, theory text in math-similar italics
\urlstyle{rm}
\isabellestyle{it}

% for uniform font size
%\renewcommand{\isastyle}{\isastyleminor}


\begin{document}

\title{Restriction\_Spaces: a Fixed-Point Theory}
\author{Benoît Ballenghien \and Benjamin Puyobro \and Burkhart Wolff}
\maketitle

\abstract{
  Fixed-point constructions are fundamental to defining recursive and co-recursive functions.
  However, a general axiom $Y f = f (Y f)$ leads to inconsistency, and definitions
  must therefore be based on theories guaranteeing existence under suitable conditions.
  In \verb'Isabelle/HOL', such constructions are typically based on sets, well-founded orders
  or domain-theoretic models such as for example \verb'HOLCF'.
  In this submission we introduce \verb'Restriction_Spaces', a formalization of spaces
  equipped with a so-called restriction, denoted by $\downarrow$,
  satifying three properties:
  \begin{align*}
    & x \downarrow 0 = y \downarrow 0\\
    & x \downarrow n \downarrow m = x \downarrow \mathrm{min} \ n \ m\\
    & x \not = y \Longrightarrow \exists n. \ x \downarrow n \not = y \downarrow n
  \end{align*}
  They turn out to be cartesian closed and admit natural notions of constructiveness and
  completeness, enabling the definition of a fixed-point operator under verifiable side-conditions.
  This is achieved in our entry, from topological definitions to induction principles.
  Additionally, we configure the simplifier so that it can automatically solve both
  constructiveness and admissibility subgoals, as long as users write higher-order rules
  for their operators.
  Since our implementation relies on axiomatic type classes, the resulting library
  is a fully abstract, flexible and reusable framework.
}


\tableofcontents

% sane default for proof documents
\parindent 0pt\parskip 0.5ex

% generated text of all theories
\input{session}

% optional bibliography
%\bibliographystyle{abbrv}
%\bibliography{root}

\end{document}

%%% Local Variables:
%%% mode: latex
%%% TeX-master: t
%%% End:
