% License: LGPL
\documentclass[10pt,a4paper]{article}
\usepackage[T1]{fontenc}
\usepackage{isabelle,isabellesym}
\usepackage{amsmath}
\usepackage{amssymb}
\usepackage{geometry}
\geometry{
  left=3cm,
  right=3cm,
  top=2cm,
  bottom=4cm
}
\usepackage{authblk}

% this should be the last package used
\usepackage{pdfsetup}

\AddToHook{cmd/section/before}{\clearpage}

% urls in roman style, theory text in math-similar italics
\urlstyle{rm}

\isabellestyle{tt}
\renewcommand{\isakeyword}[1]{\textbf{#1}}
\renewcommand{\isacommand}[1]{\isakeywordONE{#1}}
\renewcommand{\isakeywordONE}[1]{\isakeyword{\color[RGB]{0,102,153}#1}}
\renewcommand{\isakeywordTWO}[1]{\isakeyword{\color[RGB]{0,153,102}#1}}
\renewcommand{\isakeywordTHREE}[1]{\isakeyword{\color[RGB]{0,153,255}#1}}

\title{The \textbf{Weak Spectroscopy Game} \\ to Characterize Behavioral Equivalences}
\author[1,2]{Lisa A. Barthel}
\author[1]{Leonard M. Hübner}
\author[1,3]{Caroline Lemke}
\author[1]{Karl P. P. Mattes}
\author[1]{Lenard Mollenkopf}
\author[1,4]{Benjamin Bisping}
%
\affil[1]{Technische Universität Berlin, Germany}
\affil[2]{Technische Universität Ilmenau, Germany}
\affil[3]{Carl von Ossietzky Universität Oldenburg, Germany}
\affil[4]{Télécom SudParis, Institut Polytechnique de Paris, France}
\renewcommand\Affilfont{\small}
%
\begin{document}

\maketitle

\begin{abstract}
  \noindent
  We provide an Isabelle/HOL formalization of Bisping and Jansen's weak spectroscopy game~\cite{bj2024silentStepSpectroscopyExpress},
  which can be used to simultaneously characterize and decide a hierarchy of behavioral equivalences for systems with internal behavior.
  This is valuable for applications in concurrency theory and formal verification where equivalences and distinctions of the ``linear-time--branching-time spectrum'' are a recurring topic.

  This entry contains a game characterization of most behavioral equivalences from stability-respecting branching bisimilarity to weak trace equivalence.
  Technically, the results link distinguishing sublanguages of Hennessy--Milner logic to winning attacker budgets in an energy game through an eight-dimensional measurement of syntactic features appearing in formulas.
\end{abstract}

\paragraph{Overview.}
This formalization provides theoretical underpinnings of \url{https://equiv.io}, a tool to \emph{decide all behavioral equivalences at once}.
By phrasing \emph{equivalences as energy games}, one obtains a uniform way to handle a wide range of equivalences in van Glabbeek's \emph{linear-time--branching-time spectrum}~\cite{glabbeek1990ltbt1,glabbeek1993ltbt2}.
In particular, we treat systems with \emph{silent} $\tau$-steps, which usually arise because of abstraction from internal behavior, for instance, when modeling communication protocols or distributed systems using transition systems.

This formalization follows Bisping and Jansen's \emph{weak spectroscopy game}~\cite{bj2024silentStepSpectroscopyExpress}, respectively the proofs from the arXiv version~\cite{bj2023silentStepSpectroscopyArxiv}.

\begin{itemize}
  \item Section 1 provides some basics on transition systems with internal behavior.
  \item Sections 2 and 3 define a version of Hennessy--Milner logic for systems with internal behavior and a syntactic metric to select sublanguages of it through coordinates.
  \item Sections 4 to 6 prove certain coordinates to correspond to weak trace equivalence, $\eta$-bi\-si\-mi\-la\-ri\-ty, $\eta$-similarity and branching bisimilarity as well as their stable variants. (As the relationship is established through modal logics, the results can be understood as Hennessy--Milner theorems~\cite{hm1980hml}.)
  \item Sections 7 to 9 introduce the weak spectroscopy game and prove that winning attacker energies in the game correspond to coordinates of distinguishing formulas according to the syntactic expressiveness metric.
\end{itemize}

\noindent
The broader project of \emph{deciding all equivalences at once} is outlined in Bisping's PhD thesis~\cite{bisping2025generalizedEqChecking}.
There, one can also find more gentle introductions to the topic and to the game-theoretic approach in general.
The energy game approach is due to~\cite{bisping2023equivalenceEnergyGames}.

\paragraph{Acknowledgments.}
Several proofs in this document follow pen-and-paper proofs by David N. Jansen.

% do not indent paragraphs
\setlength\parindent{0pt}
\setlength\parskip{4pt}

\tableofcontents

% include generated text of all theories
\input{Labeled_Transition_Systems}
\input{LTS_Semantics}
\input{HML_SRBB}
\input{Energy}
\input{Expressiveness_Price}
\input{Weak_Traces}
\input{Eta_Bisimilarity}
\input{Branching_Bisimilarity}
\input{Energy_Games}
\input{Spectroscopy_Game}
\input{Distinction_Implies_Winning_Budgets}
\input{Strategy_Formulas}
\input{Silent_Step_Spectroscopy}

\bibliographystyle{plainurl}
\bibliography{root}

\end{document}
